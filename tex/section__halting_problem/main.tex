\section{The halting problem}

Now that we know what Turing machines are, we should discuss their capabilities and restrictions.
The Turing machine is a theoretical concept that governs modern day computers. That is, any computational problem that can be solved by a computer can also be solved by a Turing machine.
The Church-Turing thesis, as formulated for example in \cite{tur36}, further proposes that Turing machines can solve any problem that is solvable by a human being.
A problem that is not decidable is first and foremost not solvable by a Turing machine, which means there is no known way of solving the problem in an automated fashion, but it might be that we ourselves are conceptually unable to solve such a problem.

Finding a problem to be unsolvable is perhaps not the most satisfactory solution, but nonetheless a result which should probably be regarded as definitive for now.

To understand what it means for a problem to be unsolvable is the aim of this chapter.
A broader discussion on the topic may be found e.g. in \cite{sip06}.

\subsection{Decidability}

\begin{Definition}
	Let $A$ be a finite set. Then
	\begin{align*}
		A^* := \bigcup_{i \in \N_0} A^i
	\end{align*}
	is called the \emph{set of (finite) words} over the alphabet $A$.
	By $A^0$ we mean a one-element-set which we will also denote $\{\epsilon\}$, where $\epsilon \notin A$.

	A subset $L \subseteq A^*$ is called a \emph{(formal) language over $A$}.
\end{Definition}

A word over $A$ may be regarded as an initial configuration of a Turing machine $M$ on one tape on the alphabet $A$, that is the alphabet of $M$ is $A~\dot\cup~\{\EMP\}$, in the following way:
\begin{align*}
	A^* &\hookrightarrow IC(M) = (A~\dot\cup~\{\EMP\})^\Z \times S(M) \\
	(a_0, ... , a_n) &\mapsto (...\EMP,\underline{a_0},...,a_n,\EMP,...) , \INI 
\end{align*}
where $S(M)$ denotes the set of states and the underlined symbol is the one at index $0$ in $(A~\dot\cup~\{\EMP\})^\Z$.

In this context we will refer to $(a_0,...,a_n)$ as an \emph{input} to $M$.

\begin{Definition}
	\label{halting_problem:decidability:definition_decidable}
	Let $A$ be an alphabet and $L$ a formal language over $A$. Then $L$ is called \emph{decidable}, if there is a Turing machine $M$ such that $M$ stops for every $a \in A^*$ and accepts an $a \in A^*$ if $a \in L$ and rejects it otherwise.
\end{Definition}



\subsection{Undecidability}

Surely, there are lots of languages that are decidable.
A fact that might be surprising at first is that there are also languages that are not decidable.
One example for such a language is given by the halting problem.
To fully grasp this problem, we need to understand some facts about Turing machines.

First note that a Turing machine can always be finitely described. Thus there is some alphabet $A$ such that there is an embedding
\begin{align*}
	\emb{\cdot} : \{~M~|~M~\text{is a Turing machine}~\} &\hookrightarrow A^* \\
	M &\mapsto \emb{M}
\end{align*}

Similarly, we can embed other data to $A^*$, in particular, we use the following embedding which we also call \emb{\cdot}:
\begin{align*}
	\emb{\cdot} : \{ ~ (M,w) ~ | ~ M~\text{is a Turing machine}, w \in A(M)^* \} &\hookrightarrow A^* \\
	M,w &\mapsto \emb{M,w}
\end{align*}

Now we can formulate the following language:
\begin{align*}
	L_{TM} := \{~\emb{(M,w)}~|~M~\text{accepts input}~w \in A(M)^*~\}
\end{align*}

\begin{Theorem}[Halting problem]
	$L_{TM}$ is not decidable.\footnotemark
\end{Theorem}
\footnotetext{Remark \ref{remarks:scope_of_halting_problem} comments on the implications of this statement.}
\begin{proof}[Sketch of proof]
	As was already proven by Turing in \cite{tur36}, it is possible for a Turing machine on the alphabet $A$ to take some input \emb{M} and simulate the behaviour of the Turing machine $M$.
	This possibility does not depend on the choice of embedding (compare \cite{sip06}).

	Assume there was a Turing machine $M$ which decided $L_{TM}$.
	We could then formulate another Turing machine $N$ as follows:
	\begin{itemize}
		\item Check if input is of the form \emb{P}, if it is not, reject
		\item Simulate $M$ on input \emb{$~(P,\emb{P})~$}
		\item Invert the output of $M$, i.e. reject if $M$ accepts and accept if $M$ rejects
	\end{itemize}

	Now we might ask ourselves, whether $N$ accepts \emb{N}.
	As we can see, if $N$ accepts \emb{N}, then $\emb{(N,\emb{N})} \in L_{TM}$, hence $M$ accepts \emb{$~(N,\emb{N})~$} and by definition of $N$, $N$ rejects \emb{N}.
	
	But if $N$ rejects \emb{N}, then $M$ rejects \emb{$~(N,\emb{N})~$} and hence $N$ accepts \emb{N}.

	Since $M$ stops on any input, so does $N$, which means $N$ either accepts or rejects \emb{N}.
	So $N$ accepts \emb{N} if and only if $N$ rejects \emb{N}, which is a contradiction to the existence of $M$.
\end{proof}

\begin{Corollary}
	\label{halting_problem:undecidability:halting_corollary}
	The language
	\begin{align*}
		L_{TM}^\prime
		:=& \{ \emb{M} ~|~ \exists w \in A(M)^*: \emb{~(M,w)~} \in L_{TM} \} \\
		=& \{ \emb{M} ~|~ \exists w \in A(M)^* ~\text{such that $M$ accepts input $w$} \}
	\end{align*}
	is not decidable.
\end{Corollary}
\proof
	For every Turing machine $M$ and input $w$, consider the Turing machine $M_w$ which first checks whether the input is $w$, rejecting otherwise, and then simulates $M$, i.e. accepts if and only if $M$ accepts the input.

	Now assume that $L_{TM}^\prime$ was decidable. Then there would be a Turing machine which in particular decides for every $M_w$ whether it accepts some input. But this is the case if and only if $M$ accepts $w$. But this question is not decidable according to the halting problem.
\endproof


\subsection{Computability}
\label{halting_problem:computability}

Similar to the notion of decidability as presented in Definition \ref{halting_problem:decidability:definition_decidable} we can define computability:

\begin{Definition}
	\label{halting_problem:computability:definition}
	Let $X$ and $Y$ be countable sets, $A$ an alphabet and let \emb{\cdot} be embeddings from $X$ or $Y$ to $A^*$.

	Further let $f:X \to Y$ be a map.
	Then $f$ is called \emph{computable} if there is a Turing machine on the alphabet $A$ with Transition map $T$ such that
	\begin{align*}
		T^\infty(\emb{x},\INI) = (\emb{f(x)},\ACC) ~\text{for every}~ x \in X
	\end{align*}
	and
	\begin{align*}
		T^\infty(\emb{X}^c \times \{\INI\}) \subseteq \emb{Y}^c \times \{\ACC\}
	\end{align*}
\end{Definition}

\begin{Remark}
	$A^*$ is countable since it is the union of countably many finite sets.
	But we can regard $A^*$ as a subset of $(A ~\dot\cup~ \{\EMP\})^{\N_0}$ and hence $(A ~\dot\cup~ \{\EMP\})^\Z$ in a very natural way.

	If $Y$ has more than countably many elements and embeds to $(A ~\dot\cup~ \{\EMP\})^\Z$, then we define the map $f$ to be computable if there is a sequence of Turing machines with transition maps $T_i$ such that for every $x \in X$ there are $a_i \in A^*$ such that
	\begin{align*}
		T_i^\infty(\emb{x}, \INI) = (a_i,\ACC) ~\text{for every}~ i \in \N
	\end{align*}
	and $a_i \xrightarrow{i \to \infty} \emb{f(x)}$.

	Equivalently we could say that there is a single Turing machine such that the tape-part of $T^i(\emb{x},\INI)$ up to translation of the tape converges to $\emb{f(x)}$.
\end{Remark}

In the following chapters we prove some problems to be undecidable.
This means that there is a way of embedding instances of that problem into the words over some alphabet and no matter which embedding we choose there is no Turing machine that accepts all instances of the problem with positive answer and rejects all instances with negative answer.

To do so, we will manipulate a multitude of mathematical objects and usually do so in a computable way, making sure that in the end there will be a Turing machine which translates instances of our original problem to instances of the halting problem.
Then we can claim that if there was a Turing machine that solves our problem, then there would also be a Turing machine which solves the halting problem, leading to a contradiction.
To be more precise, let us formulate the following lemma:

\begin{Lemma}
	\label{halting_problem:computability:lemma_decidability}
	Let $L$ be an undecidable language
	and $M$ be a language such that there is a computable map $f$ from $L$ to $M$.

	Then $M$ is undecidable.
\end{Lemma}
\begin{proof}
	Let $Q$ be a Turing machine which computes $f$.

	Assume there was a Turing machine $S$ which decided $M$.
	Then $S \circ Q$ would decide $L$.
	To see this, let $l \in L \cup L^c$. Then 
	\begin{align*}
		T_{state}^\infty(S \circ Q)(l) = T_{state}^\infty(S) (T^\infty(Q)(l)) = \begin{cases} \ACC &~\text{if}~ l \in L \\ \REJ &~\text{if}~ l \in L^c\end{cases}
	\end{align*}
	(Compare Remark \ref{turing_machines:basic_notions:remark_concatenation}.)
	But such a Turing machine must not exist as $L$ is undecidable, leading to a contradiction to the existence of $S$.
\end{proof}

There is a problem with rigorously checking computability in every instance, since that would probably overshadow everything else done in this work.
Therefore we resolve to a more intuitive notion of computability.\footnotemark~
A nice description of this notion can be found in \cite{bbj07}:
\footnotetext{Remark \ref{remarks:intuitive_algorithms} tells us where one might get such an intuition from.}
\begin{quotation}
	``A function $f$ from positive integers to positive integers is called \emph{effectively computable} if a list of instructions can be given that in principle make it possible to determine the value $f(n)$ for any argument $n$. The instructions must be completely definite and explicit. They should tell you at each step what to do, not tell you to go ask someone else what to do, or figure out for yourself what to do: the instructions should require no external sources of information, and should require no ingenuity to execute, so that one might hope to automate the process of applying the rules, and have it performed by some mechanical device.''
\end{quotation}
This basically boils down to the slogan
\begin{quotation}
	\textbf{``Constructive proofs yield computable results''}
\end{quotation}
Of course we have to be extra careful when employing such a notion and while reading this work we should always convince ourselves that all results are actually computable whenever necessary.

