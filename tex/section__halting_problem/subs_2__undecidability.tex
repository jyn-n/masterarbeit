\subsection{Undecidability}

Surely, there are lots of languages that are decidable.
A fact that might be surprising at first is that there are also languages that are not decidable.
One example for such a language is given by the halting problem.
To fully grasp this problem, we need to understand some facts about Turing machines.

First note that a Turing machine can always be finitely described. Thus there is some alphabet $A$ such that there is an embedding
\begin{align*}
	\emb{\cdot} : \{~M~|~M~\text{is a Turing machine}~\} &\hookrightarrow A^* \\
	M &\mapsto \emb{M}
\end{align*}

Similarly, we can embed other data to $A^*$, in particular, we use the following embedding which we also call \emb{\cdot}:
\begin{align*}
	\emb{\cdot} : \{ ~ (M,w) ~ | ~ M~\text{is a Turing machine}, w \in A(M)^* \} &\hookrightarrow A^* \\
	M,w &\mapsto \emb{M,w}
\end{align*}

Now we can formulate the following language:
\begin{align*}
	L_{TM} := \{~\emb{(M,w)}~|~M~\text{accepts input}~w \in A(M)^*~\}
\end{align*}

\begin{Theorem}[Halting problem]
	$L_{TM}$ is not decidable.\footnotemark
\end{Theorem}
\footnotetext{Remark \ref{remarks:scope_of_halting_problem} comments on the implications of this statement.}
\begin{proof}[Sketch of proof]
	As was already proven by Turing in \cite{tur36}, it is possible for a Turing machine on the alphabet $A$ to take some input \emb{M} and simulate the behaviour of the Turing machine $M$.
	This possibility does not depend on the choice of embedding (compare \cite{sip06}).

	Assume there was a Turing machine $M$ which decided $L_{TM}$.
	We could then formulate another Turing machine $N$ as follows:
	\begin{itemize}
		\item Check if input is of the form \emb{P}, if it is not, reject
		\item Simulate $M$ on input \emb{$~(P,\emb{P})~$}
		\item Invert the output of $M$, i.e. reject if $M$ accepts and accept if $M$ rejects
	\end{itemize}

	Now we might ask ourselves, whether $N$ accepts \emb{N}.
	As we can see, if $N$ accepts \emb{N}, then $\emb{(N,\emb{N})} \in L_{TM}$, hence $M$ accepts \emb{$~(N,\emb{N})~$} and by definition of $N$, $N$ rejects \emb{N}.
	
	But if $N$ rejects \emb{N}, then $M$ rejects \emb{$~(N,\emb{N})~$} and hence $N$ accepts \emb{N}.

	Since $M$ stops on any input, so does $N$, which means $N$ either accepts or rejects \emb{N}.
	So $N$ accepts \emb{N} if and only if $N$ rejects \emb{N}, which is a contradiction to the existence of $M$.
\end{proof}

\begin{Corollary}
	The language
	\begin{align*}
		L_{TM}^\prime
		:=& \{ \emb{M} ~|~ \exists w \in A(M)^*: \emb{~(M,w)~} \in L_{TM} \} \\
		=& \{ \emb{M} ~|~ \exists w \in A(M)^* ~\text{such that $M$ accepts input $w$} \}
	\end{align*}
	is not decidable.
\end{Corollary}
\proof
	For every Turing machine $M$ and input $w$, consider the Turing machine $M_w$ which first checks whether the input is $w$, rejecting otherwise, and then simulates $M$, i.e. accepts if and only if $M$ accepts the input.

	Now assume that $L_{TM}^\prime$ was decidable. Then there would be a Turing machine which in particular decides for every $M_w$ whether it accepts some input. But this is the case if and only if $M$ accepts $w$. But this question is not decidable according to the halting problem.
\endproof
