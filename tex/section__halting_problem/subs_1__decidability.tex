\subsection{Decidability}

\begin{Definition}
	Let $A$ be a finite set. Then
	\begin{align*}
		A^* := \bigcup_{i \in \N_0} A^i
	\end{align*}
	is called the \emph{set of (finite) words} over the alphabet $A$.
	By $A^0$ we mean a one-element-set which we will also denote $\{\epsilon\}$, where $\epsilon \notin A$.

	A subset $L \subseteq A^*$ is called a \emph{(formal) language over $A$}.
\end{Definition}

A word over $A$ may be regarded as an initial configuration of a Turing machine $M$ on one tape on the alphabet $A$, that is the alphabet of $M$ is $A~\dot\cup~\{\EMP\}$, in the following way:
\begin{align*}
	A^* &\hookrightarrow IC(M) = (A~\dot\cup~\{\EMP\})^\Z \times S(M) \\
	(a_0, ... , a_n) &\mapsto (...\EMP,\underline{a_0},...,a_n,\EMP,...) , \INI 
\end{align*}
where $S(M)$ denotes the set of states and the underlined symbol is the one at index $0$ in $(A~\dot\cup~\{\EMP\})^\Z$.

In this context we will refer to $(a_0,...,a_n)$ as an \emph{input} to $M$.

\begin{Definition}
	\label{halting_problem:decidability:definition_decidable}
	Let $A$ be an alphabet and $L$ a formal language over $A$. Then $L$ is called \emph{decidable}, if there is a Turing machine $M$ such that $M$ stops for every $a \in A^*$ and accepts an $a \in A^*$ if $a \in L$ and rejects it otherwise.
\end{Definition}

