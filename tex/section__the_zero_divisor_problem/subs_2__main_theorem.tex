\subsection{Trivial \ltwo-betti-numbers}
\label{the_zero_divisor_problem:main_theorem}

Now we are ready to understand and prove Theorem \ref{zero-divisor-problem:trivial_betti_numbers} which may be considered the most important Theorem of this work.
For this proof, we will need Theorem 4.3 in conjunction with Lemma 6.4 from \cite{gra14}:

\begin{Theorem}
	\label{the_zero_divisor_problem:grabowskis_theorem}
	Let
	\begin{itemize}
		\item{
			$M$ be a Turing dynamical system that
			\begin{itemize}
				\item is foolproof
				\item does not restart
				\item has disjoint accepting chains
			\end{itemize}
		}
		\item{
			$(X, \mu)$ be the associated probability measure space where
			\begin{itemize}
				\item $X$ has an added group structure
				\item which is of the form $\prod_{\N} \Z_2$
				\item the $X_i$ are also of that form up to isomorphy
				\item $\mu$ is the haar measure
			\end{itemize}
		}
		\item $\Gamma$ be the group acting by $\rho$ on $X$ as defined by $M$ which should act by continuous group automorphisms\footnotemark
		\item $I$ be the initial set
		\item $\Omega_2$ be the second fundamental value
	\end{itemize}

	Then there is an algorithm that produces an element $S \in \Q[\widehat{X} \rtimes_{\widehat{\rho}} \Gamma]$ such that
	\begin{align*}
		\dim_{vN} ~ \ker ~ S = \mu(I) - \Omega_2
	\end{align*}
\end{Theorem}

\footnotetext{Remark \ref{remarks:continuous_group_automorphisms} comments on continuous group automorphisms in our construction of a TDS in chapter \ref{tm_to_tds:construction}.}

Now let us have a look at the following computational problem, which we call \emph{trivial \ltwo-betti-numbers}:

\begin{Problem}
	Given a discrete, countable group $G$, a free, finite-type $G$-CW-Complex $X$ and $n \in \N$, is the n'th \ltwo-betti-number of $X$ equal to 0?
\end{Problem}

\begin{Theorem}
	\label{zero-divisor-problem:trivial_betti_numbers}
	\emph{Trivial \ltwo-betti-numbers} is not decidable for \IM{G = (\Z_2 \wr \Z)^2}.
\end{Theorem}

%TODO zusammenhang zw kernen und bettizahlen

To prove this, we first consider a similar problem, namely:

\begin{Problem}
	Given a discrete, countable group G and a matrix \IM{M \in (\Z[G])^{k \times k}}
	and considering the multiplication with $M$ in $\ltwo(G)^k$, is \IM{\ker M} trivial?
\end{Problem}
%TODO left or right multiplication

This problem we call \emph{kernel-over-$\Z[G]$}.

\begin{Theorem}
	\label{zero-divisor-problem:kernel-over-zg}
	\emph{Kernel-over-$\Z[G]$} is not decidable for \IM{G = (\Z_2 \wr \Z)^2}.
\end{Theorem}

\proof
	We do a reduction to the halting problem by following steps as can be seen in Figure \ref{zero_divisor_problem:main_theorem:fig_proof_plan}.
	We transform the objects on the left hand side in a way that preserves the property on the right hand side.

	\begin{figure}[h]
		\centering
		\begin{tikzpicture}[shorten >=1pt,on grid,auto]
	\node[] (0) {Turing machine};
	\node[] (1) [below=2 of 0] {foolproof readonly Turing machine};
	\node[] (2) [below=2 of 1] {Turing dynamical system};
	\node[] (5) [below=2 of 2] {$(\Z_2 \wr \Z)^2$-CW-Complex};
	
	\node[] (10) [right=8 of 0] {accepts some input};
	\node[] (11) [right=8 of 1] {accepts some input};
	\node[] (12) [right=8 of 2] {accepts more than a null set};
	\node[] (15) [right=8 of 5] {trivial \ltwo-betti-number};

	\path[->]
		(0) edge (1)
		(1) edge (2)
		(2) edge (5)
	;
\end{tikzpicture}

		\caption{Steps of this proof}
		\label{zero_divisor_problem:main_theorem:fig_proof_plan}
	\end{figure}

	So, let $M$ be a Turing machine on one tape.
	The halting problem states that the question whether there is some input which $M$ accepts is undecidable.
	Theroem \ref{turing_machines:main_theorem:theorem} gives us a foolproof readonly Turing machine $M^\prime$ on two tapes such that there is some input which $M$ accepts if and only if there is some input which $M$ accepts.\footnotemark
	\footnotetext{The statement remains true for Turing machines on multiple tapes but Theorem \ref{turing_machines:main_theorem:theorem} does not provide this.}
	Swapping the meaning of accepting and rejecting we may instead assume that $M^\prime$ rejects some input if and only if there is some input which $M$ accepts.

	Using the construction from chapter \ref{tm_to_tds} we get a Turing dynamical system that
	does not restart,
	is foolproof (Lemma \ref{tm_to_tds:properties:lemma_foolproof})
	has disjoint accepting chains (Lemma \ref{tm_to_tds:properties:lemma_disjoint_accepting_chains})
	thus having $\Omega_1 = \Omega_2$ (Lemma \ref{tds:lemma_disj_acc_ch:lemma}) and
	has a fundamental value strictly smaller than $\mu(I)$ if and only if there is some input which $M$ accepts (Lemma \ref{tm_to_tds:properties:lemma_fundamental_value}).
	The space of configurations of this TDS is $X = ((\Z_2^k)^\Z)^2 \times (\Z_2)^l$ for some $k, l \in \N$
	and the acting group $\Gamma$ is $\Z^2 \times \Aut \Z_2^l$
	where $\Z^2$ acts on the index set of the first factor of $X$ by addition
	and $\Aut \Z_2^l$ acts on $(\Z_2)^l$ in the natural way.
	Let us call these operations $\rho$ and $\sigma$ respectively.

	Theorem \ref{the_zero_divisor_problem:grabowskis_theorem} now states that there is an element $S \in \Q[\widehat{((\Z_2^k)^\Z)^2 \times \Z_2^l} \rtimes_{\widehat{\rho \times \sigma}} (\Z^2 \times \Aut \Z_2^l)$] that can be produced in a computable fashion such that $\dim_{vN} \ker S = \mu(I) - \Omega_2$.
	This value however is $0$ if and only if there is some input which $M$ accepts. So the question whether $\dim_{vN} \ker S = 0$ is generally undecidable.

	It remains to show that we can asssociate $S$ to an element $N \in \Z[(\Z_2 \wr \Z)^2]^{k \times k}$ such that $\dim_{vN} \ker N = 0 \Leftrightarrow \dim_{vN} \ker S = 0$.

%TODO really widehat, ref semidirect product
	First we multiply $S$ by a sufficiently large integer such that it is an element of the group ring over $\Z$. We then have
	\begin{align*}
		      & \Z [ (\widehat{((\Z_2^k)^\Z)^2 \times \Z_2^l}) \rtimes_{\widehat{\rho \times \sigma}} (\Z^2 \times \Aut \Z_2^l) ]
		\\
		\cong~& \Z [ ( \widehat{((\Z_2^k)^\Z)^2} \rtimes_{\widehat{\rho}} \Z^2 ) \times ( \widehat{\Z_2^l} \rtimes_{\widehat\sigma} \Aut \Z_2^l ) ]
			~\text{(Lemma \ref{the_zero_divisor_problem:pontryagin_duality:lemma_product_of_dual_group_actions})} \\
		\cong~& \Z [ ( \widehat{((\Z_2^k)^\Z)^2} \rtimes_{\widehat{\rho}} \Z^2 ) ] \otimes \Z [ ( \widehat{\Z_2^l} \rtimes_{\widehat\sigma} \Aut \Z_2^l ) ]
		\\
		\cong~& \Z [ ( \widehat{(\Z_2^k)^\Z} \rtimes_{\widehat{\rho}} \Z )^2 ] \otimes \Z [ ( \widehat{\Z_2^l} \rtimes_{\widehat\sigma} \Aut \Z_2^l ) ]
			~\text{(Lemma \ref{the_zero_divisor_problem:pontryagin_duality:lemma_dual_product})}
		\\
		\cong~& \Z [ ( (\Z_2^k)^{\oplus\Z} \rtimes_{\rho} \Z )^2 ] \otimes \Z [ ( \widehat{\Z_2^l} \rtimes_{\widehat\sigma} \Aut \Z_2^l ) ]
			~\text{(Examples \ref{the_zero_divisor_problem:pontryagin_duality:example_duals} and \ref{the_zero_divisor_problem:pontryagin_duality:example_dual_action})}
	\end{align*}
	Here $\rho$ generally denotes the action of $\Z$ by addition on the index set for several groups.

	Now consider the group $(\Z_2^k)^{\oplus\Z}$. It is isomorphic to $\Z_2^{\oplus \Z}$ by
	\begin{align*}
		& ((x_{i,j})_{i = 1 .. k})_{j \in \Z} \mapsto (x_{k \cdot j + i})_{i \in \{1,..,k\}, j \in \Z}
	\end{align*}
	and transforming $\rho$ via this automorphism yields
	\begin{align*}
		(\Z_2^k)^{\oplus\Z} \rtimes_\rho \Z ~\cong~ \Z_2^{\oplus\Z} \rtimes_{\rho} \Z ~\cong~ \Z_2 \wr \Z
	\end{align*}
	and thus we may assume $S \in \Z[(\Z_2 \wr \Z)^2] \otimes \Z[H]$ for some finite group $H$ (namely $\widehat{\Z_2^l} \rtimes_{\widehat\sigma} \Aut \Z_2^l$).
	But for a finite group $H$, $\ltwo(H) \cong \C[H]$ and thus an element of $\Z[H]$, which is a linear map on $\ltwo(H)$, can be seen as a finitely dimensioned matrix with entries in $\Z$.
	Hence $\Z[H] \hookrightarrow \Z^{n \times n}$ in a way that translates von-Neumann-Trace to the usual trace of a matrix (up to a factor $n$).
	To see this note that the von-Neumann trace of an element of $(a_g)_{g \in G} \in \Z[H]$ is just $a_1$, which translates to every diagonal element of the matrix.
%TODO erklaeren

	In particular, a $0$-von-Neumann trace translates to a $0$-trace of the matrix.

	But now we have
	\begin{align*}
		S ~\in~ \Z[(\Z_2 \wr \Z)^2] \otimes \Z[H] ~\subseteq~ \Z[(\Z_2 \wr \Z)^2] \otimes \Z^{n \times n} ~\cong~ \Z[(\Z_2 \wr \Z)^2]^{n \times n}
	\end{align*}
	and a $0$-von-Neumann trace translates to a $0$-trace, which concludes the proof.
\endproof
%TODO erklaeren dass zu jeder TM eine Matrix gehoert

\proof[Proof of Theorem \ref{zero-divisor-problem:trivial_betti_numbers}]
	A free, finite type $G$-CW-complex can be described uniquely by its attaching maps which in turn can be described as a finite dimensional matrix with entries in $\Z[G]$. Also given such a matrix we can construct a unique $G$-CW-complex. The \ltwo-homology then corresponds to the kernel of that matrix.\footnotemark

	Thus if there is an algorithm that solves \emph{trivial \ltwo-betti-numbers} for a given group $G$ then that algorithm also solves \emph{kernel-over-$\Z[G]$}. But Theorem \ref{zero-divisor-problem:kernel-over-zg} states that there is no such algorithm for $G = (\Z_2 \wr \Z)^2$.
	\footnotetext{For a full proof of these statements refer to \cite{luc02}.}
\endproof

