\subsection{Pontryagin duality}

\begin{Definition}
	\label{the_zero_divisor_problem:pontryagin_duality:definition_pontryagin_duality}
	Let $G$ be a locally compact abelian group. Then we define the \emph{pontryagin-dual group} as
	\begin{align*}
		\widehat{G} := \Hom(G,\T)
	\end{align*}
	where $\T$ is the circle group, that is the subgroup of $(\C,\cdot)$ that contains all numbers on the unit circle, and $\Hom$ means the morphisms in the category of topological groups, i.e. continuous group homomorphisms.
\end{Definition}

\begin{Lemma}
	If $G$ is finite, then so is $\widehat{G}$.
\end{Lemma}

\proof
	For every $\varphi \in \Hom(G,\T)$ and $g \in G$, the order of $g$ has to be a multiple of the order of $\varphi(g)$.
	For any $k \in \N$, there are only finitely many elements of order $k$ in $\T$.
	Therefore for every $g \in G$, there are only finitely many choices for the value of $\varphi(g)$.
	Finiteness of $G$ completes the proof.
\endproof

\begin{Lemma}
	\label{the_zero_divisor_problem:pontryagin_duality:lemma_dual_product}
	Let $I$ be an index set and $G_i$ a group for every $i \in I$ such that $\widehat{\prod_{i \in I} G_i}$ is defined. Then
	\begin{align*}
		\widehat{\prod_{i \in I} G_i} \cong \bigoplus_{i \in I} \widehat{G_i}
	\end{align*}
\end{Lemma}

\proof
	Let
	\begin{align*}
		\pr_i: \prod_{j \in I} G_j \to G_i
	\end{align*}
	be the canonical projection onto the $i$'th component and
	\begin{align*}
		\iota_i: G_i \hookrightarrow \prod_{j \in I} G_j
	\end{align*}
	the canonical embedding.

	Let $\varphi \in \widehat{\prod_{i \in I} G_i}$. Then we have
	\begin{align*}
		\varphi = \prod_{i \in I} \varphi \circ \iota_i \circ \pr_i
	\end{align*}
	where the product refers to the multiplication in $\C$.
	But for this to be well defined we require that almost all $\varphi \circ \iota_i \circ \pr_i$ are null.
	Thus the isomorphism we are looking for is $\varphi \mapsto (\varphi \circ \iota_i)_{i \in I}$ and the inverse is $(\psi_i)_{i \in I} \mapsto \prod_{i \in I} \psi_i \circ \pr_i$.
\endproof

\begin{Example}
	\label{the_zero_divisor_problem:pontryagin_duality:example_duals}
	\
	\begin{itemize}
		\item {$\widehat{\Z_2} = \Hom(\Z_2,\T) \cong \Z_2$}
		\item {$\widehat{(\Z_2)^n} \cong (\Z_2)^n ~\text{for all}~ n \in \N$}
		\item {$\widehat{(\Z_2)^\Z} \cong (\Z_2)^{\oplus \Z}$}
	\end{itemize}
\end{Example}

\begin{Definition}
	Let $\rho: \Gamma \curvearrowright G$ be a group action.
	Then we define the {pontryagin-dual group action} by the commutativity of the following diagram:
	\begin{figure}[H]
		\centering
		\begin{tikzpicture}[shorten >=1pt,on grid,auto]
	\node[] (0) {$\Gamma$};
	\node[] (1) [right=4 of 0] {$\Aut(G)$};
	\node[] (2) [right=4 of 1] {$\Aut(G)$};
	\node[] (3) [below=3 of 2] {$\Aut(\widehat{G})$};

	\path[->]
		(0) edge [ above ]
			node { $\rho$ }
			(1)
		(1) edge [ above ]
			node { $\cdot^{-1}$ }
			(2)
		(2) edge [ right ]
			node { $\widehat{\cdot}$ }
			(3)
		(0) edge [ above ]
			node { $\widehat{\rho}$ }
			(3)
	;
\end{tikzpicture}


	\end{figure}
	Here, $\widehat{\cdot}$ refers to the application of the $\Hom$-functor, regarding an element of $Aut(G)$ as a morphism in the category of topological groups.
\end{Definition}
\begin{Remark}
	We have
	\begin{align*}
		\widehat{\rho}(\gamma)(\varphi) = \varphi \circ \rho(\gamma^{-1})
	\end{align*}
\end{Remark}
\begin{Example}
	\label{the_zero_divisor_problem:pontryagin_duality:example_dual_action}
	Consider $\rho: \Z \curvearrowright (\Z_2)^\Z$ acting by addition on the index set.
	Then $\widehat{\rho}$ is the action of $\Z$ by addition on the index set of $(\Z_2)^{\oplus \Z}$ and vice-versa.
\end{Example}

\begin{Lemma}
	\label{the_zero_divisor_problem:pontryagin_duality:lemma_product_of_dual_group_actions}
	Let $\rho: \Gamma \curvearrowright G$ and $\sigma: \Delta \curvearrowright H$ be group actions. Then
	\begin{align*}
		(\widehat{G \times H}) \rtimes_{\widehat{\rho \times \sigma}} (\Gamma \times \Delta) ~\cong~ (\widehat{G} \times \widehat{H}) \rtimes_{\widehat{\rho} \times \widehat{\sigma}} (\Gamma \times \Delta)
	\end{align*}
\end{Lemma}

\proof
	Consider the following maps:
	\begin{figure}[H]
		\centering
		\begin{tikzpicture}[shorten >=1pt,on grid,auto]
	\node[] (0) {$\Gamma \times \Delta$};
	\node[] (1) [below=2 of 0] {$\Aut \widehat{G} \times \Aut \widehat{H}$};
	\node[] (2) [below=2 of 1] {$\Aut(\widehat{G} \times \widehat{H})$};
	\node[] (3) [below=2 of 2] {$\Aut(\widehat{G \times H})$};

	\node[] (10) [right=6 of 0] {$\gamma , \delta$};
	\node[] (11) [right=6 of 1] {$\lambda \mapsto (g \mapsto \lambda(\gamma^{-1} g)), \mu \mapsto (h \mapsto \mu(\delta^{-1} h))$};
	\node[] (12) [right=6 of 2] {$(\lambda,\mu) \mapsto (g \mapsto \lambda(\gamma^{-1} g), h \mapsto \mu(\delta^{-1} h))$};
	\node[] (13) [right=6 of 3] {$\lambda \mapsto ((g,h) \mapsto \lambda(\gamma^{-1} g, \delta^{-1} h))$};

	\path[->]
		(0) edge [ right ]
			node {$\widehat{\varphi} \times \widehat{\psi}$}
			(1)
		(1) edge 
			node {$\iota$}
			(2)
		(2) edge
			node {$\iota$}
			(3)
		(0) edge [ bend right=90 ]
			node {$\widehat{\varphi \times \psi}$}
			(3)
	;
	\path[|->]
		(10) edge (11)
		(11) edge (12)
		(12) edge (13)
		(10) edge [ bend right=90 ] (13)
	;
\end{tikzpicture}


	\end{figure}
	Regarding $\Aut{\widehat{G}} \times \Aut{\widehat{H}}$ as a subset of $\Aut \widehat{G \times H}$ we have $\widehat{\rho \times \sigma} = \widehat{\rho} \times \widehat{\sigma}$ which concludes the proof.
\endproof

