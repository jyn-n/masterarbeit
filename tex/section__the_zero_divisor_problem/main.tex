\section{The Zero-divisor problem}

\subsection{Pontryagin duality}

\begin{Definition}
	Let $G$ be a locally compact abelian group. Then we define the \emph{pontryagin-dual group} as
	\begin{align*}
		\widehat{G} := \Hom(G,\T)
	\end{align*}
	where $\T$ is the circle group, that is the subgroup of $(\C,\cdot)$ that contains all numbers on the unit circle, and $\Hom$ means the morphisms in the category of topological groups, i.e. continuous group homomorphisms.
\end{Definition}

\begin{Lemma}
	If $G$ is finite, then so is $\widehat{G}$.
\end{Lemma}

\proof
	For every $\varphi \in \Hom(G,\T)$ and $g \in G$, the order of $g$ has to be a multiple of the order of $\varphi(g)$.
	For any $k \in \N$, there are only finitely many elements of order $k$ in $\T$.
	Therefore for every $g \in G$, there are only finitely many choices for the value of $\varphi(g)$.
	Finiteness of $G$ completes the proof.
\endproof

\begin{Lemma}
	\label{the_zero_divisor_problem:pontryagin_duality:lemma_dual_product}
	Let $I$ be an index set and $G_i$ a group for every $i \in I$. Then
	\begin{align*}
		\widehat{\prod_{i \in I} G_i} \cong \bigoplus_{i \in I} \widehat{G_i}
	\end{align*}
\end{Lemma}

\proof
	Let
	\begin{align*}
		\pr_i: \prod_{i \in I} G_i \to G_i
	\end{align*}
	be the projection onto the $i$'th component and
	\begin{align*}
		\iota_i: G_i \to \prod_{i \in I} G_i
	\end{align*}
	the canonical embedding.

	Let $\varphi \in \widehat{\prod_{i \in I} G_i}$. Then we have
	$\varphi = \prod_{i \in I} \varphi \circ \iota_i \circ \pr_i$.
	But for this to be well defined we require that almost all $\varphi \circ \iota_i \circ \pr_i$ are null.
	Thus the isomorphism we are looking for is $\varphi \mapsto (\varphi \circ \iota_i)_{i \in I}$ and the inverse is $(\psi_i)_{i \in I} \mapsto \prod_{i \in I} \psi_i$.
\endproof

\begin{Example}
	\label{the_zero_divisor_problem:pontryagin_duality:example_duals}
	\
	\begin{itemize}
		\item {$\widehat{\Z_2} \cong \Z_2$}
		\item {$\widehat{(\Z_2)^n} \cong (\Z_2)^n ~\forall n \in \N$}
		\item {$\widehat{(\Z_2)^\Z} \cong (\Z_2)^{\oplus \Z}$}
	\end{itemize}
\end{Example}

\begin{Definition}
	Let $\rho: \Gamma \curvearrowright G$ be a group action.
	Then we define the {pontryagin-dual group action} by the commutativity of the following diagram:
	\begin{figure}[H]
		\centering
		\begin{tikzpicture}[shorten >=1pt,on grid,auto]
	\node[] (0) {$\Gamma$};
	\node[] (1) [right=4 of 0] {$\Aut(G)$};
	\node[] (2) [right=4 of 1] {$\Aut(G)$};
	\node[] (3) [below=3 of 2] {$\Aut(\widehat{G})$};

	\path[->]
		(0) edge [ above ]
			node { $\rho$ }
			(1)
		(1) edge [ above ]
			node { $\cdot^{-1}$ }
			(2)
		(2) edge [ right ]
			node { $\widehat{\cdot}$ }
			(3)
		(0) edge [ above ]
			node { $\widehat{\rho}$ }
			(3)
	;
\end{tikzpicture}


	\end{figure}
	Here, $\widehat{\cdot}$ refers to the application of the $\Hom$-functor, regarding an element of $Aut(G)$ as a morphism in the category of topological groups.
\end{Definition}

\begin{Example}
	\label{the_zero_divisor_problem:pontryagin_duality:example_dual_action}
	Consider $\rho: \Z \curvearrowright (\Z_2)^\Z$ acting by addition on the index set.
	Then $\widehat{\rho}$ is the action of $\Z$ by addition on the index set of $(\Z_2)^{\oplus \Z}$.
\end{Example}

\begin{Lemma}
	\label{the_zero_divisor_problem:pontryagin_duality:lemma_product_of_dual_group_actions}
	Let $\rho: \Gamma \curvearrowright G$ and $\sigma: \Delta \curvearrowright H$ be group actions. Then
	\begin{align*}
		(\widehat{G \times H}) \rtimes_{\widehat{\rho \times \sigma}} (\Gamma \times \Delta) ~\cong~ (\widehat{G} \times \widehat{H}) \rtimes_{\widehat{\rho} \times \widehat{\sigma}} (\Gamma \times \Delta)
	\end{align*}
\end{Lemma}

\proof
	Consider the following maps:
	\begin{figure}[H]
		\centering
		\begin{tikzpicture}[shorten >=1pt,on grid,auto]
	\node[] (0) {$\Gamma \times \Delta$};
	\node[] (1) [below=2 of 0] {$\Aut \widehat{G} \times \Aut \widehat{H}$};
	\node[] (2) [below=2 of 1] {$\Aut(\widehat{G} \times \widehat{H})$};
	\node[] (3) [below=2 of 2] {$\Aut(\widehat{G \times H})$};

	\node[] (10) [right=6 of 0] {$\gamma , \delta$};
	\node[] (11) [right=6 of 1] {$\lambda \mapsto (g \mapsto \lambda(\gamma^{-1} g)), \mu \mapsto (h \mapsto \mu(\delta^{-1} h))$};
	\node[] (12) [right=6 of 2] {$(\lambda,\mu) \mapsto (g \mapsto \lambda(\gamma^{-1} g), h \mapsto \mu(\delta^{-1} h))$};
	\node[] (13) [right=6 of 3] {$\lambda \mapsto ((g,h) \mapsto \lambda(\gamma^{-1} g, \delta^{-1} h))$};

	\path[->]
		(0) edge [ right ]
			node {$\widehat{\varphi} \times \widehat{\psi}$}
			(1)
		(1) edge 
			node {$\iota$}
			(2)
		(2) edge
			node {$\iota$}
			(3)
		(0) edge [ bend right=90 ]
			node {$\widehat{\varphi \times \psi}$}
			(3)
	;
	\path[|->]
		(10) edge (11)
		(11) edge (12)
		(12) edge (13)
		(10) edge [ bend right=90 ] (13)
	;
\end{tikzpicture}


	\end{figure}
	Regarding $\Aut{\widehat{G}} \times \Aut{\widehat{H}}$ as a subset of $\Aut \widehat{G \times H}$ we have $\widehat{\rho \times \sigma} = \widehat{\rho} \times \widehat{\sigma}$ which concludes the proof.
\endproof


\subsection{}

From \cite{gra14} we quote Theorem 4.3 in conjunction with Lemma 6.4

\begin{Theorem}
	\label{the_zero_divisor_problem:grabowskis_theorem}
	Let
	\begin{itemize}
		\item{
			$M$ be a Turing dynamical system that
			\begin{itemize}
				\item{is foolproof}
				\item{does not restart}
				\item{has disjoint accepting chains}
			\end{itemize}
		}
		\item{
			$(X, \mu)$ the associated probability measure space where
			\begin{itemize}
				\item{$X$ has an added group structure}
				\item{which is ofthe form $\prod \Z_2$}
				\item{the $X_i$ are also of that form}
			\end{itemize}
		}
		\item{$\Gamma$ the group acting by $\rho$ on $X$ as defined by $M$ which should act by continuous group automorphisms}
		\item{$I$ the initial set}
		\item{$\Omega_2$ the second fundamental value}
	\end{itemize}

	Then there is an algorithm that produces an element $S \in \Q[\widehat{X} \rtimes_{\widehat{\rho}} \Gamma]$ such that
	\begin{align*}
		\dim_{vN} ~ \ker ~ S = \mu(I) - \Omega_2
	\end{align*}
\end{Theorem}

Now let us have a look at the following computational problem, which we call \emph{trivial $l^2$-betti-numbers}:

Given a group $G$, a $G$-CW-Complex $X$ and $n \in \N$, is the n'th $l^2$-betti-number of $X$ equal to 0?

\begin{Theorem}
	\label{zero-divisor-problem:trivial_betti_numbers}
	\emph{Trivial $l^2$-betti-numbers} is not decidable for \IM{G = (\Z_2 \wr \Z)^2}.
\end{Theorem}

To prove this, we first consider a similar problem, namely:

Given a group G and a matrix \IM{M \in (\Z[G])^{k \times k}}
and considering the multiplication with $M$ in $l^2(G)^k$, is \IM{\ker M} trivial?

This problem we call \emph{kernel-over-$\Z[G]$}.

\begin{Theorem}
	\label{zero-divisor-problem:kernel-over-zg}
	\emph{Kernel-over-$\Z[G]$} is not decidable for \IM{G = (\Z_2 \wr \Z)^2}.
\end{Theorem}

\proof
	We do a reduction to the halting problem. To this end, let $M$ be a Turing machine on one tape.
	The halting problem states that the question whether there is some input which $M$ accepts is undecidable.
	Theroem \ref{turing_machines:main_theorem:theorem} gives us a foolproof readonly Turing machine $M^\prime$ on two tapes such that there is some input which $M$ accepts if and only if there is some input which $M$ accepts.\footnotemark
	\footnotetext{The statement remains true for Turing machines on multiple tapes but Theorem \ref{turing_machines:main_theorem:theorem} does not provide this.}
	Swapping the meaning of accepting and rejecting we may instead assume that $M^\prime$ rejects some input if and only if there is some input which $M$ accepts.

	Using the construction from chapter \ref{tm_to_tds} we get a Turing dynamical system that
	does not restart,
	is foolproof (Lemma \ref{tm_to_tds:properties:lemma_foolproof})
	has disjoint accepting chains (Lemma \ref{tm_to_tds:properties:lemma_disjoint_accepting_chains})
	thus having $\Omega_1 = \Omega_2$ (Lemma \ref{foo}) and
	%TODO lemma about fundamental values when disj acc chains
	has a fundamental value strictly smaller than $\mu(I)$ if and only if there is some input which $M$ accepts (Lemma \ref{tm_to_tds:properties:lemma_fundamental_value}).
	The space of configurations of this TDS is $X = ((\Z_2^k)^\Z)^2 \times (\Z_2)^l$ for some $k, l \in \N$
	and the acting group $\Gamma$ is $\Z^2 \times \Aut \Z_2^l$
	where $\Z^2$ acts on the index set of the first factor of $X$ by addition
	and $\Aut \Z_2^l$ acts on $(\Z_2)^l$.
	Let us call these operations $\rho$ and $\sigma$ respectively.

	Theorem \ref{the_zero_divisor_problem:grabowskis_theorem} now states that there is an element $S \in \Q[\widehat{((\Z_2^k)^\Z)^2} \rtimes_{\widehat{\rho \times \sigma}} (\Z^2 \times \Aut \Z_2^l)$] that can be produced in an algorithmical fashion such that $\dim_{vN} \ker S = \mu(I) - \Omega_2$.
	This value however is $0$ if and only if there is some input which $M$ accepts. So the question whether $\dim_{vN} \ker S = 0$ is generally undecidable.

	It remains to show that we can asssociate $S$ to an element $N \in \Z[(\Z_2 \wr \Z)^2]^{k \times k}$ such that $\dim_{vN} \ker N = 0 \Leftrightarrow \dim_{vN} \ker S = 0$.

	First we multiply $S$ by a sufficiently large integer such that it is an element of the group ring over $\Z$. We then have
	\begin{align*}
		      & \Z [ (\widehat{((\Z_2^k)^\Z)^2 \times \Z_2^l}) \rtimes_{\widehat{\rho \times \sigma}} (\Z^2 \times \Aut \Z_2^l) ]
		\\
		\cong~& \Z [ ( \widehat{((\Z_2^k)^\Z)^2} \rtimes_{\widehat{\rho}} \Z^2 ) \times ( \widehat{\Z_2^l} \rtimes_{\widehat\sigma} \Aut \Z_2^l ) ]
			~\text{(Lemma \ref{the_zero_divisor_problem:pontryagin_duality:lemma_product_of_dual_group_actions})} \\
		\cong~& \Z [ ( \widehat{((\Z_2^k)^\Z)^2} \rtimes_{\widehat{\rho}} \Z^2 ) ] \otimes \Z [ ( \widehat{\Z_2^l} \rtimes_{\widehat\sigma} \Aut \Z_2^l ) ]
		\\
		\cong~& \Z [ ( \widehat{(\Z_2^k)^\Z} \rtimes_{\widehat{\rho}} \Z )^2 ] \otimes \Z [ ( \widehat{\Z_2^l} \rtimes_{\widehat\sigma} \Aut \Z_2^l ) ]
			~\text{(Lemma \ref{the_zero_divisor_problem:pontryagin_duality:lemma_dual_product})}
		\\
		\cong~& \Z [ ( (\Z_2^k)^{\oplus\Z} \rtimes_{\rho} \Z )^2 ] \otimes \Z [ ( \widehat{\Z_2^l} \rtimes_{\widehat\sigma} \Aut \Z_2^l ) ]
			~\text{(Examples \ref{the_zero_divisor_problem:pontryagin_duality:example_duals} and \ref{the_zero_divisor_problem:pontryagin_duality:example_dual_action} \footnotemark)}
	\end{align*}
%TODO see footnote
	\footnotetext{uhm. to be clean I actually need that groups are naturally isomorphic to their bidual since the second example is the wrong way around...}
	Here $\rho$ generally denotes the action of $\Z$ by addition on the index set for several groups.

	Now consider the group $(\Z_2^k)^{\oplus\Z}$. It is isomorphic to $\Z_2^{\oplus \Z}$ by
	\begin{align*}
		& ((x_{i,j})_{i = 1 .. k})_{j \in \Z} \mapsto (x_{k \cdot j + i})_{k \in \{1,..,k\}, j \in \Z}
	\end{align*}
	and transforming $\rho$ via this automorphism yields
	\begin{align*}
		(\Z_2^k)^{\oplus\Z} \rtimes_\rho \Z \cong \Z_2^{\oplus\Z} \rtimes_{\rho} \Z \cong \Z_2 \wr \Z
	\end{align*}
	and thus we may assume $S \in \Z[(\Z_2 \wr \Z)^2] \otimes \Z[H]$ for some finite group $H$ (namely $\widehat{\Z_2^l} \rtimes_{\widehat\sigma} \Aut \Z_2^l$).
	But for a finite group $H$, $l^2(H) \cong \C[H]$ and thus an element of $\Z[H]$, which is a linear map on $l^2(H)$, can be seen as a finitely dimensioned matrix with entries in $\Z$.
	Hence $\Z[H] \hookrightarrow \Z^{n \times n}$ in a way that translates von-Neumann-Trace to the usual trace of a matrix (up to a factor $n$).
	To see this note that the von-Neumann trace of an element of $(a_g)_{g \in G} \in \Z[H]$ is just $a_1$, which translates to every diagonal element of the matrix.

	In particular, a $0$-von-Neumann trace translates to a $0$-trace of the matrix.

	But now we have (up to (von-Neumann-) trace preserving isomorphisms)
	\begin{align*}
		S ~\in~ \Z[(\Z_2 \wr \Z)^2] \otimes \Z[H] ~\subseteq~ \Z[(\Z_2 \wr \Z)^2] \otimes \Z^{n \times n} ~\cong~ \Z[(\Z_2 \wr \Z)^2]^{n \times n}
	\end{align*}
	which concludes the proof.
\endproof

\proof \emph{of Theorem \ref{zero-divisor-problem:trivial_betti_numbers}}
	A finite type $G$-CW-complex can be described uniquely by its attaching maps which in turn can be described as a finite dimensional matrix with entries in $\Z[G]$. Also given such a matrix we can construct a unique $G$-CW-complex. The $l^2$-cohomology then corresponds to the kernel of that matrix.
	Thus if there is an algorithm that solves \emph{trivial $l^2$-betti-numbers} for a given group $G$ then that algorithm also solves \emph{kernel-over-$\Z[G]$}. But Theorem \ref{zero-divisor-problem:kernel-over-zg} states that there is no such algorithm for $G = \Z_2 \wr \Z$.
\endproof

