\section{Decidable Zero-divisors}

%TODO ganz viele Texte

\begin{Definition}
	Let $G$ be a group which is finitely presented by $g_1,...,g_n$. This means that any element of $G$ can be presented as a word over the alphabet $\{g_1,g_1^{-1},...,g_n,g_n^{-1}\}$. The \emph{word problem (for groups)} now asks whether a given word represents the unit element of $G$.. That is, for
	\begin{align*}
		v_1,...,v_k \in \{g_1,g_1^{-1},...,g_n,g_n^{-1}\}
	\end{align*}
	is $v_1 \cdot ... \cdot v_k = 1$?
\end{Definition}

This problem is not decidable in general. %TODO citation needed

\begin{Example}
	The group $Z_2 \wr \Z$ is generated by
	\begin{align*}
		x &:= (1_0,0) ~\text{and} \\
		y &:= (0,1)
	\end{align*}
	where $1_i \in \Z_2^\Z$ is the element such that
	\begin{align*}
		 (1_i)_j = \begin{cases} 1~ &\text{, if $i = j$} \\ 0~ &\text{, otherwise} \end{cases}
	\end{align*}
	The word problem for this presentation is decidable.
\end{Example}
\proof
	Let $v_1,...,v_n \in \{x,y,y^{-1}\}$.

	Clearly, $v_1 \cdot ... \cdot v_n$ is of the form $(*,0)$, if and only if $y$ and $y^{-1}$ occur the same number of times among the $v_i$.\footnotemark
	\footnotetext{This actually means solving the word problem for $\Z$. But for free groups and for groups with a single generator, the word problem clearly is decidable and $\Z$ is both.}

	For the first component, we can regard and element of $\Z_2^{\oplus \Z}$ as a finite subset of $\Z$.
	Then the action of $\Z$ (which is generated by $y$) can be seen as a (computable!) bijection on $\Z$.
	Thus deciding whether the first component of $v_1 \cdot ... \cdot v_n$ is zero boils down to computing the exclusive disjunction of some finite sets, which clearly is doable.
\endproof

\begin{Definition}
	A group $G$ is called a \emph{BFS-group}, if there is a bound in the size of finite subgroups of $G$.
	That is there is an $n \in N$ such that for every $H \leq G$ we have $|H| \leq n$ or $|H| = \infty$
\end{Definition}

\begin{Example}
	$\Z_2 \wr \Z$ is not a BFS-group.
\end{Example}
\proof
	$\Z_2 \wr \Z$ contains $\Z_2^{\oplus \Z}$ as a subgroup, which in turn contains $\Z_2^n$ for every $n \in \N$. But $|\Z_2^n| = 2^n$ which is finite and larger than $n$ for every $n \in \N$.
\endproof

\begin{Theorem}
	\label{decidable_zero_divisors:theorem}
	Let $G$ be a finitely presented sofic BFS-group with decidable word problem for which the Atiyah-conjecture holds.%TODO Atiyah, sofic
	Then \emph{kernel-over-$\Z[G]$} (compare chapter \ref{the_zero_divisor_problem:main_theorem}) is decidable.
\end{Theorem}
For the proof, let us quote Lemma 7 from \cite{gra14-2}:
\begin{Lemma}
		\label{decidable_zero_divisors:lemma}
	Let $G$ be a sofic group.

	Then there is a computable map $h: \N_+ \to \N_+$ such that for every self-adjoint, positive $T \in \Z[G]$ such that at most $n$ coefficients are non-zero and each coefficient is less than $n$, then
	\begin{align*}
		| \dim_{vN} \ker T - \tr_{vN}(1 - \frac{T}{\|T\|_1})^{h(n)} | < \frac{1}{n}
	\end{align*}
	where $\|T\|_1$ is the sum of the absolutes of all coefficients.
\end{Lemma}
\proof [Proof of Theorem \ref{decidable_zero_divisors:theorem}]
	Let $T \in \Z[G]$.
	The Atiyah-conjecture provides a $k \in \N$ such that $\dim_{vN} T \in \{0, \frac{1}{k},\frac{2}{k},...\}$. %TODO is k computable? Has to come from BFS
	In particular, $\dim_{vN} T \neq 0 \Rightarrow \dim_{vN} T \geq \frac1k$.

	Now consider the map $f:k \mapsto \tr_{vN}(1 - \frac{T}{\|T\|_1})^{h(3k)}$ ($h$ from Lemma \ref{decidable_zero_divisors:lemma}).
	This map is computable since in order to compute the von-Neumann-trace, we only need to compute the coefficients of $f(k)$, which means we only need to solve the word problem for $G$, which is possible as per requirements of the theorem.

	By Lemma \ref{decidable_zero_divisors:lemma} we have $|f(k) - \dim_{vN} \ker T| < \frac{1}{3k}$. But this means
	\begin{align*}
		&T ~\text{is a zero-divisor in \Z[G]} \\
		\iff~& \dim_{vN} \ker T \neq 0 \\
		\iff~& \dim_{vN} \ker T \geq \frac1k \\
		\iff~& \dim_{vN} f(k) > \frac{1}{3k}
	\end{align*}
	which is decidable. For the last step, realize that if $f(k) \leq \frac{1}{3k}$, then $|f(k) - \frac1k| \geq \frac{1}{3k}$.
\endproof
