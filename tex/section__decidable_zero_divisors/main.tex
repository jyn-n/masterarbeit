\section{Decidable Zero-divisors}
\label{decidable_zero_divisors}

As we have seen, \emph{kernel-over-$\Z[(\Z_2 \wr \Z)^2]$} is not decidable.
To put this statement into context, we will consider a class of groups for which \emph{kernel-over-$\Z[G]$} actually is decidable and see which properties $(\Z_2 \wr \Z)^2$ is lacking.
Hence we learn something about the sharpness of said class of groups.

To understand these groups we need to briefly introduce some properties: First we consider the word problem, which gives us a link to the decidability of problems:

\begin{Definition}
	Let $G$ be a finitely presented group with generators $g_1,...,g_n$.
	The \emph{word problem (for groups)} now asks whether a given word represents the unit element of $G$. That is, for
	\begin{align*}
		v_1,...,v_k \in \{g_1,g_1^{-1},...,g_n,g_n^{-1}\}
	\end{align*}
	is $v_1 \cdot ... \cdot v_k = 1$?
\end{Definition}

\begin{Remark}
	\
	\begin{itemize}
		\item The word problem is not decidable in general. %TODO citation needed
		\item If the word problem is decidable for some presentation of a group $G$, then it is decidable for any presentation of $G$.
			Hence we may say that the word problem is or is not decidable for $G$.
	\end{itemize}
\end{Remark}

\begin{Example}
	The group $\Z_2 \wr \Z$ is generated by
	\begin{align*}
		x &:= (1_0,0) ~\text{and} \\
		y &:= (0,1)
	\end{align*}
	where $1_i \in \Z_2^\Z$ is the element such that
	\begin{align*}
		 (1_i)_j = \begin{cases} 1~ &\text{, if $i = j$} \\ 0~ &\text{, otherwise} \end{cases}
	\end{align*}

	To see this remember that the multiplication in $\Z_2 \wr \Z ~\cong~ \Z_2^{\oplus \Z} \rtimes \Z$ is defined by
	\begin{align*}
		(a,b) \cdot (1_i,d) = (a + 1_{b+i}, b+d)
	\end{align*}
	for $b,d \in \Z$ and $a \in \Z_2^{\oplus \Z}$.

	The word problem for this group is decidable.
\end{Example}
\proof
	Let $v_1,...,v_n \in \{x,y,y^{-1}\}$.

	Clearly, $v_1 \cdot ... \cdot v_n$ is of the form $(*,0)$, if and only if $y$ and $y^{-1}$ occur the same number of times among the $v_i$.\footnotemark
	\footnotetext{This actually means solving the word problem for $\Z$. But for free groups as well as cyclic groups, the word problem clearly is decidable and $\Z$ is both.}

	For the first component, note that $\Z_2^\Z$ may be considered to be the tape of a Turing machine.
	If we take $0$ to be the \EMP-symbol, we have in fact that every element of $\Z_2^{\oplus \Z}$ corresponds to a finite tape (that is, when we add a state, it corresponds to a finite configuration).

	To compute the first component, consider a Turing machine on two tapes, which takes $v_1,...,v_n$ as input on the first tape and writes the first component of their product onto the second tape by using the identification as explained above.\footnotemark~
	To do so, the Turing machine should traverse the $v_i$ from left to right.
	Whenever it encounters an $x$, it toggles the symbol on the second tape.
	And for $y$ or $y^{-1}$, it shifts the second tape by one symbol to the left or right.
	\footnotetext{In fact, this Turing machine uses different alphabets on each tape which is technically not allowed, but this is no problem as formally, we might say that the common alphabet is the disjoint union or the cartesian product of the alphabets of each tape.}

	By marking every position on the second tape that was ever encountered (compare chapter \ref{turing_machines:main_theorem}), we get a consecutive area of marked positions.
	Unmarked positions are known to be $0$.
	Therefore, we may easily check if after the last $v_i$ was handled, there is any $1$ on the second tape. If there is none, then the product of the $v_i$ is $0$ and vice-versa.
\endproof

\begin{Definition}
	A group $G$ is called a \emph{BFS-group}, if there is a bound in the size of finite subgroups of $G$.
	That is there is an $n \in \N$ such that for every $H \leq G$ we have $|H| \leq n$ or $|H| = \infty$.
\end{Definition}

\begin{Remark}
	A BFS-group is in a sense the next best thing to a torsion free group.
	In fact, a group is torsion free, if and only if the size of its finite subgroups is bounded by 1.
\end{Remark}

\begin{Example}
	$\Z_2 \wr \Z$ is not a BFS-group.
\end{Example}
\proof
	$\Z_2 \wr \Z$ contains $\Z_2^{\oplus \Z}$ as a subgroup, which in turn contains $\Z_2^n$ for every $n \in \N$. But $|\Z_2^n| = 2^n$ which is finite and larger than $n$ for every $n \in \N$.
\endproof

Conjecturally, for every BFS-group $G$, there is a $k \in \N$ such that every \ltwo-betti-number of $G$ is a multiple of $\frac1k$.
This is known as the Atiyah-conjecture for BFS-groups.\footnotemark
\footnotetext{Remark \ref{remarks:atiyah-conjecture} comments on the original Atiyah-conjecture.}

The following theorem gives us some groups for which \emph{kernel-over-$\Z[G]$} is decidable.
However, as we have learned, $(\Z_2 \wr \Z)^2$ is not such a group.
In particular, there are non-BFS-groups, which meet the other requirements of the theorem, such that \emph{kernel-over-$\Z[G]$} is not decidable.

\begin{Theorem}
	\label{decidable_zero_divisors:theorem}
	Let $G$ be a Group which
	\begin{itemize}
		\item is finitely presented
		\item is sofic
		\item is BFS
		\item has decidable word problem
		\item fulfills the Atiyah-conjecture for BFS-groups
	\end{itemize}

	Then \emph{kernel-over-$\Z[G]$} (compare chapter \ref{the_zero_divisor_problem:main_theorem}) is decidable.
\end{Theorem}
For the proof, let us quote Lemma 7 from \cite{gra14-2}:
\begin{Lemma}
		\label{decidable_zero_divisors:lemma_sofic}
	Let $G$ be a sofic group.

	Then there is a computable map $h: \N_+ \to \N_+$ such that for every
	\begin{itemize}
		\item self-adjoint, positive $T \in \Z[G]$
		\item and $n \in \N$ such that
		\begin{itemize}
			\item at most $n$ coefficients in $T$ are non-zero
			\item and each coefficient is absolutely less than $n$
		\end{itemize}
	\end{itemize}
	then
	\begin{align*}
		| \dim_{vN} \ker T - \tr_{vN}(1 - \frac{T}{\|T\|_1})^{h(n)} | < \frac{1}{n}
	\end{align*}
	where $\|T\|_1$ is the sum of the absolutes of all coefficients.
\end{Lemma}
We omit the proof of this lemma (for that, refer to \cite{gra14-2}) but let us note, that $\dim_{vN} \ker T$ is defined to be the von-Neumann trace of a projection onto $\ker T$.
The lemma now states that $(1 - \frac{T}{\|T\|_1})^{h(n)}$ is almost such a projection. That is, applying this map multiple times brings us arbitraryly close to $\ker T$.

Also note that the eigenvalues of $\frac{T}{\|T\|_1}$ are at most $1$, which makes the statement quite believable at least in case of a finite group $G$.
The infinite case then is provided by soficity.

We need this lemma since we cannot compute the von-Neumann dimension as we probably do not know any projection onto the kernel.

Also we will need the following lemma:
\begin{Lemma}
	\label{decidable_zero_divisors:lemma_word_problem}
	Let $G$ be a group
	and $S,T \in \Z[G]$.

	Then the coefficients of $S \cdot T$ are computable if and only if the word problem for $G$ is decidable.
\end{Lemma}
\proof
	Let the coefficients of $S$ be denoted by $S = (S_g)_{g \in G}$ and similar for $T$ and $S \cdot T$.
	Then
	\begin{align*}
		(S \cdot T)_g = \sum_{h \in G} S_{gh^{-1}} T_h
	\end{align*}
	This means that in order to compute $(S \cdot T)_g$, given two presentations of group elements $h_1$ and $h_2$, we have to decide if $h_1 h_2 = g$ which is indeed equivalent to the word problem.
\endproof

\proof [Proof of Theorem \ref{decidable_zero_divisors:theorem}]
	Let $T \in \Z[G]$.
	The Atiyah-conjecture provides a $k \in \N$ such that $\dim_{vN} \ker T \in \{0, \frac{1}{k},\frac{2}{k},...\}$. %TODO is k computable? Has to come from BFS
	In particular, $\dim_{vN} \ker T \neq 0 \Rightarrow \dim_{vN} \ker T \geq \frac1k$.

	Now consider the map $f:\N \to \R, m \mapsto \tr_{vN}(1 - \frac{T}{\|T\|_1})^{h(3m)}$ ($h$ from Lemma \ref{decidable_zero_divisors:lemma_sofic}).
	This map is computable since in order to compute the von-Neumann-trace, we only need to compute the coefficients of $(1 - \frac{T}{\|T\|_1})^{h(3m)}$, which means we only need to solve the word problem for $G$ (compare lemma \ref{decidable_zero_divisors:lemma_word_problem}), which is possible as per requirements of the theorem.

	By Lemma \ref{decidable_zero_divisors:lemma_sofic} we have $|f(k) - \dim_{vN} \ker T| < \frac{1}{3k}$. But this means
	\begin{align*}
		& \dim_{vN} \ker T \neq 0 \\
		\iff~& \dim_{vN} \ker T \geq \frac1k \\
		\iff~& f(k) > \frac{1}{3k}
	\end{align*}
	which is decidable. For the last step, realize that if $f(k) \leq \frac{1}{3k}$, then $|f(k) - \frac1k| \geq \frac{1}{3k}$.
\endproof
