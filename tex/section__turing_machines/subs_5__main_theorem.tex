To conclude this chapter we prove a theorem about Turing machines which allows us to use certain classes of Turing machines interchangeably, a feature which becomes important later on.
\begin{Theorem}
	For every Turing machine $M$ on one tape (i.e. $n(M) = 1$), there is a foolproof readonly Turing machine $N$ on two tapes such that
	there is a computable map $f: IC(M) \to C(N)$
	satisfying \IM{f(AC(M)) = AC(N)}.
\end{Theorem}
\proof
	Let $M$ be a Turing machine on one tape.
	We first construct the map $f$ to get an idea what $N$ is supposed to do and then construct $N$ itself.
	But to define $f$ we have to know what $C(N)$ looks like.
	The state-part of $f(c)$ should always be \INI. Therefore it suffices for now to define $A(N)$:
	\begin{align*}
		A(N) := (A(M) \times \{0,1\} \times \{0,1\})~\dot\cup~S(M)~\dot\cup~\{|\}~\dot\cup~\{\EMP\}
	\end{align*}
	Think of a symbol $a \in A(N)$ as either a state or a symbol of $M$, or the ``separator'' symbol $|$.
	If $a$ happens to be a symbol of $M$ it might or might not be ``marked'' in two different ways.
	The first marking should be interpreted as ``position $0$'', the second marking means ``new position''
	and for $a \in A(M)$ we will write
	$\underline{a}$, $\boldsymbol{a}$ and $\underline{\boldsymbol{a}}$ for position $0$-$a$, new position $a$ and new position $0$-$a$ (i.e. $(a,1,0)$, $(a,0,1)$ and $(a,1,1)$) respectively.

	Thus a finite configuration $c \in FC(M)$ may be interpreted as a word over this alphabet:
	The first symbol is the state of $c$. The following symbols are the (mostly) non-\EMP-part of the tape with the symbol at index $0$ underlined.

	Similarly a word over $A(N)$ might denote a sequence of configurations by containing each of those configurations separated by $|$.

	Now we define the tape-part of $f(c)$ to be the word
	\begin{align*}
		c~|~T(c)~|~T^2(c)~|~...~|~T^k(c)
	\end{align*}
	if $T^k(c) = T^{k+1}(c)$ and the respective infinitely long word otherwise (note that every single configuration is still finite).

	There is some ambiguity in this definition as it is not clear exactly how many \EMP-symbols should be in a given configuration.
	This we resolve thus: $c$ itself should be as short as possible, that is it contains the underlined symbol, all non-\EMP-symbols and any \EMP in between, but no more.
	Two consecutive configurations should have the same length if possible. If not, the second configuration may be one symbol longer.
	This occurs if the underlined symbol is rightmost (or leftmost) and $T_{direction}$ is $1$ (or $-1$).
	In this and only this case, the underlined symbol in the second configuration should be bold (i.e. marked as new position).

\begin{tikzpicture}[shorten >=1pt,on grid,auto]
	\node[state] (0) {\INI};
	\node[state] (1) [below=6 of 0] {2};
	\node[state] (2) [right=6 of 1] {3};
	\node[state] (3) [below=6 of 2] {4};
	\node[state] (4) [left=6 of 3] {1};
	\node[state] (6) [below=6 of 4] {\ACC};
	\node[state] (5) [right=12 of 6] {\textbf{4}};

	\path[->]
		(0) edge [ left ]
			node {$\begin{matrix}s \\ s\end{matrix}, s \in S(M)$ }
			(1)
		(1) edge [ above ]
			node {$\begin{matrix}x \\ x\end{matrix}, x \in A(M)$ }
			(2)
		(2) edge [ loop , in=15 , out=75 , distance=50 , above ]
			node {$\begin{matrix}x \\ x\end{matrix}, x \in A(M)$}
			(2)
		(2) edge [ right ]
			node {$\begin{matrix}\underline{x} \\ \underline{x}\end{matrix}, x \in A(M)$}
			(3)
		(3) edge [ loop , in=285 , out=345 , distance=50 , below ]
			node {$\begin{matrix}x \\ x\end{matrix}, x \in A(M)$}
			(3)
		(3) edge [ below ]
			node {$\begin{matrix}| \\ |\end{matrix}$}
			(4)
		(4) edge [ left ]
			node {$\begin{matrix}\EMP \\ \EMP\end{matrix}$}
			(6)
		(4) edge [ left ]
			node {$\begin{matrix}s \\ s\end{matrix}, s \in S(M)$}
			(1)
		(2) edge [ bend left ]
			node {$\begin{matrix}\underline{\boldsymbol{x}} \\ \underline{\boldsymbol{x}} \end{matrix}, x \in A(M)$}
			(5)
		(5) edge [ bend left ]
			node {$\begin{matrix}| \\ |\end{matrix}$}
			(4)
		(1) edge [ bend left ]
			node {$\begin{matrix}\underline{x} \\ \underline{x}\end{matrix}, x \in A(M)$}
			(3)
		(1) edge [ bend right ]
			node {$\begin{matrix}\underline{\boldsymbol{x}} \\ \underline{\boldsymbol{x}}\end{matrix}, x \in A(M)$}
			(3)
	;
\end{tikzpicture}

\newpage
\begin{tikzpicture}[shorten >=1pt,on grid,auto]
	\node[state] (0) {\INI};
	\node[state] (1) [below=4 of 0] {-1};
	\node[state] (2) [below=4 of 1] {0};
	\node[state] (3) [below=4 of 2] {1};
	\node[state] (4) [right=8.5 of 3] {2};
	\node[state] (5) [above=7 of 4] {3};
	\node[state] (6) [right=5.5 of 4] {4};
	\node[state] (7) [below=7 of 4] {5};
	\node[state] (8) [below=7 of 3] {\ACC};

	\path[->]
		(0) edge [ left ]
			node {$\begin{matrix}s \\ s\end{matrix} \mapsto \begin{matrix}0 \\ 1\end{matrix}$ }
			(1)
		(1) edge [ loop left ]
			node {$\begin{matrix}* \\ *\end{matrix} \mapsto \begin{matrix}0 \\ 1\end{matrix}$}
			(1)
		(1) edge [ left ]
			node {$\begin{matrix}* \\ \underline{a}\end{matrix} \mapsto \begin{matrix}0 \\ 1\end{matrix},(a,*)$}
			(2)
		(2) edge [ loop left ]
			node {$\begin{matrix}* \\ *\end{matrix} \mapsto \begin{matrix}0 \\ 1\end{matrix}$}
			(2)
		(2) edge [ left ]
			node {$\begin{matrix}* \\ |\end{matrix} \mapsto \begin{matrix}0 \\ 1\end{matrix}$}
			(3)
		(3) edge [ left ]
			node {$\begin{matrix}\ACC \\ *\end{matrix}$}
			(8)
		(3) edge [ bend left , blue ]
			node {$(a,*),\begin{matrix}s \\ T_{state}(a,s)\end{matrix} \mapsto \begin{matrix}1 \\ 1\end{matrix},(a,s)$}
			(4)
		(4) edge [ bend left , green ]
			node {$\begin{matrix}| \\ |\end{matrix} \mapsto \begin{matrix}1 \\ 1\end{matrix}$}
			(3)
		(4) edge [ loop right ]
			node {$\begin{matrix}x \\ x\end{matrix} \mapsto \begin{matrix}1 \\ 1\end{matrix}$}
			(4)
		(4) edge [ loop left , red ]
			node {$\begin{matrix}
				(a,s),\begin{matrix}\underline{a} \\ \underline{T_{write}(a,s)}\end{matrix} \mapsto \begin{matrix}1 \\ 1\end{matrix},(T_{write}(a,s),s) \\
				\text{if}~T_{direction}(a,s) = 0
			\end{matrix}$}
			(4)
		(4) edge [ bend left , near end , red ]
			node {$\begin{matrix}
				(a,s),\begin{matrix}x \\ \underline{x}\end{matrix} \mapsto \begin{matrix}1 \\ 1\end{matrix},(x,s) \\
				\text{if}~T_{direction}(a,s) = -1
			\end{matrix}$}
			(5)
		(5) edge [ bend left , near start ]
			node {$(x,s),\begin{matrix}\underline{a} \\ T_{write}(a,s)\end{matrix} \mapsto \begin{matrix}1 \\ 1\end{matrix},(x,s)$}
			(4)
		(4) edge [ bend left , red ]
			node {$\begin{matrix}
				(a,s),\begin{matrix}\underline{a} \\ \underline{\boldsymbol{x}}\end{matrix} \mapsto \begin{matrix}0 \\ 1\end{matrix},(x,s) \\
				\text{if}~T_{direction}(a,s) = -1
			\end{matrix}$}
			(6)
		(6) edge [ bend left ]
			node {$(x,s),\begin{matrix}\underline{a} \\ T_{write}(a,s)\end{matrix} \mapsto \begin{matrix}1 \\ 1\end{matrix},(x,s)$}
			(4)
		(4) edge [ bend left , near end , red ]
			node {$\begin{matrix}
				(a,s),\begin{matrix}\underline{a} \\ T_{write}(a,s)\end{matrix} \mapsto \begin{matrix}1 \\ 1 \end{matrix}, (a,s) \\
				\text{if}~T_{direction}(a,s) = 1
			\end{matrix}$}
			(7)
		(7) edge [ bend left=10 , near start ]
			node {$\begin{matrix}x \\ \underline{x}\end{matrix} \mapsto \begin{matrix}1 \\ 1\end{matrix}$}
			(4)
		(7) edge [ bend left ]
			node {$\begin{matrix}| \\ \underline{\boldsymbol{x}}\end{matrix} \mapsto \begin{matrix}0 \\ 1\end{matrix}$}
			(4)
	;
\end{tikzpicture}


\endproof
