\section{Turing machines} \label{turing_machines}

\subsection{Definition}

\begin{Definition}
	A \emph{Turing machine} $M$ consists of the following data:
	\begin{itemize}
		\item{some $n \in \N$}, the \emph{number of tapes}
		\item{a finite set $A$ called the \emph{alphabet}, its elements are called \emph{symbols}}
		\item{a finite set $S$ called the \emph{states}}
		\item{a map $T_{write}: A^n \times S \to A^n$}
		\item{a map $T_{state}: A^n \times S \to S$}
		\item{a map $T_{direction}: A^n \times S \to D^n$ where $D := \{-1,0,1\} \subseteq \Z$}
		\item{distinct states $\{\INI,\ACC,\REJ\} \subseteq S$}
		\item{a symbol $\EMP \in A$}
	\end{itemize}
	and satisfies
	\begin{itemize}
		\item{$\INI \notin T_{state}(A^n \times S)$}
		\item{$(T_{write} \times T_{state} \times T_{direction}) (x,s) = (x,s,0) ~\text{for every}~ x \in A^n , s \in \STOPS$}
	\end{itemize}
\end{Definition}
When talking about Turing machines we will always use the denominations as in the definition.
For example the alphabet will always be called $A$ and $A$ will usually be the alphabet of some Turing machine.
If it is not obvious to which Turing machine $A$ belongs, we clarify by writing $A(M)$.

An element $((a_{i,j})_{i \in \Z, j \in \{1,...,n\}},s) \in (A^\Z)^n \times S =: C$ is called a \emph{configuration} of $M$. The $(a_{*,j}) \in A^\Z$ are called the \emph{tapes}.
For improved readibility we will leave out the indices of $a$ whenever possible.

A configuration is said to be \emph{finite}, if only finitely many entries on each tape are not \EMP.

A configuration $(a,s) \in C$ is called an \emph{initial configuration}, if all of the following are true:
\begin{itemize}
	\item{$(a,s)$ is finite}
	\item{$s = \INI$}
	\item{$a_{z,i} = \EMP~\forall z < 0, i \in \{1,...,n\}$}
\end{itemize}
The sets of finite and initial configurations are denoted by $FC$ and $IC$ respectively.

We will now define the \emph{transition map} which shall be a map on the configurations $C$ which in a sense combines the behaviour of $T_{write}$, $T_{state}$ and $T_{direction}$.
To this end we first extend the domains of $T_{write}$, $T_{state}$ and $T_{direction}$ to $C$:
Let
\begin{align*}
	\pr_0: (A^\Z)^n &\to A^n \\
	(\pr_0(a))_i &:= a_{0,i}~\forall i \in \{1,...,n\}
\end{align*}
Now define
\begin{align*}
	T_{write}^\prime &:= T_{write} \circ (\pr_0 \times \id_S) \\
	T_{state}^\prime &:= T_{state} \circ (\pr_0 \times \id_S) \\
	T_{direction}^\prime &:= T_{direction} \circ (\pr_0 \times \id_S)
\end{align*}
Further let
\begin{align*}
	T_{write}^{\prime\prime}: (A^\Z)^n \times S &\to (A^\Z)^n \\
	(a,s) &\mapsto T_{write}^{\prime\prime}(a,s) \\
\end{align*}
where
\begin{align*}
	(T_{write}^{\prime\prime}(a,s))_{z,i} =
	\begin{cases}
		(T_{write}^\prime(a,s))_i &\text{, if}~z = 0 \\
		a_{z,i} &\text{, if}~z \neq 0
	\end{cases}
\end{align*}
As it should become clear from the context which map is being referred to, we will just say $T_{write}$ instead of $T_{write}^{\prime\prime}$
and $T_{state}$ instead of $T_{state}^{\prime}$
and the same for $T_{direction}$.

Further we require the map $T_{tape}$:
Given some tapes and a state, it regards the symbols at position 0 of every tape and changes them according to the map $T_{write}$.
Then it shifts those new tapes according to $T_{direction}$ but to determine the direction it uses the unchanged symbols.

More precisely, let
\begin{align*}
	\sh: A^\Z \times \Z &\to A^\Z \\
	(a,z) &\mapsto \sh(a,z)
\end{align*}
where
\begin{align*}
	\sh_i(a,z) = a_{i+z}
\end{align*}
Now let
\begin{align*}
	T_{shift}: (A^\Z)^n \times C &\to (A^\Z)^n \\
	(b,c) &\mapsto ( \sh(b_{*,1},(T_{direction}(c))_1) , ... , \sh(b_{*,n},(T_{direction}(c))_n) )
\end{align*}
and
\begin{align*}
	T_{tape}: C &\to (A^\Z)^n \\
	c &\mapsto T_{shift} (T_{write} (c) , c)
\end{align*}

Finally we can define the \emph{transition map} $T:C\to C$ nicely:
\begin{align*}
	T = T_{tape} \times T_{state}
\end{align*}
\begin{Remark}
	In the literature one often finds Turing machines to be defined in a way such that configurations are always finite. In our case, the broader definition will allow us to easily adjust to the concept of Turing dynamical systems later on.
	But nevertheless our Turing machines still satisfy $T(FC) \subseteq FC$, that is, we can restrict our definition to finite configurations and still have a well defined notion of Turing machines, which even happens to coincide with the one from standard literature.
\end{Remark}

If for some $c \in C$ and $k \in \N$ we have that $T^k(c) = T^{k+1}(c)$ we define $T^\infty(c) := T^k(c)$ and leave it undefined otherwise.

We also abuse notation a bit and define
\begin{align*}
	T_{state}^0: C &\to S \\
	(a,s) &\mapsto s
\end{align*}
as well as
\begin{align*}
	T_{tape}^0: C &\to (A^\Z)^n \\
	(a,s) &\mapsto a
\end{align*}
and also
\begin{align*}
	T_{tape}^k := T_{tape}^0 \circ T^k
\end{align*}
and
\begin{align*}
	T_{state}^k := T_{state}^0 \circ T^k
\end{align*}
for all $k \in \N \cup \{\infty\}$.
To justify this notation, observe that $T^k = T_{tape}^k \times T_{state}^k$.


A Turing machine is said to
\begin{itemize}
	\item{\emph{accept $c$}, if $T_{state}^\infty(c)$ is defined and equals \ACC}
	\item{\emph{reject $c$}, if $T_{state}^\infty(c)$ is defined and equals \REJ}
	\item{\emph{stop for $c$}, if it accepts $c$ or rejects $c$}
	\item{\emph{stop}, if it stops for any initial configuration}
	\item{be \emph{foolproof}, if it stops for any finite configuration}
	\item{be \emph{readonly}, if $T_{write}(a,s) = a~\forall a \in A^n, s \in S$}
\end{itemize}
Of a readonly Turing machine we will further require that it does not ``cross'' the \EMP-symbol, that is for every $N \in \N$ and $a \in (A^\Z)^n$ the following holds:
\begin{align*}
	a_{l,i} \neq \EMP~\forall~l \in \{0,...,(\sum_{j=1}^N T_{direction}^j(a,s)_i) -1\}
\end{align*}
if the sum is positive and analogous if the sum is negative.\footnotemark

\footnotetext{Using the notion of similarity from Theorem \ref{turing_machines:main_theorem:theorem} we can see that this is not a real restriction.}

The subsets of $IC$ that $M$ accepts, rejects and stops for will be denoted by $AC$, $RC$ and $SC$ respectively.

\begin{Remark}
	$M$ stops for $c$ if and only if there is some $k \in \N_0$ such that \IM{T_{state}^k(c) \in \STOPS}.
\end{Remark}

Now let us present a method of turning multiple Turing machines into one by ``first executing one, then the other one''. This will be useful later as it gives us a method to divide a large Turing machine into more managable parts.

\begin{Definition}
	Let $M$ and $N$ be Turing machines such that $A(M) = A(N)$ and $n(M) = n(N)$, that is, their configurations differ only by the states. Then we define their \emph{concatenation} $N \circ M$ as the Turing machine defined by the following data:
	\begin{itemize}
		\item{$n = n(M) = n(N)$}
		\item{$A = A(M) = A(N)$}
		\item{$\EMP = \EMP(M) = \EMP(N)$}
		\item{$S = S(M)~\dot\cup~S(N)$}
		\item{$\INI = \INI(M)$}
		\item{$\ACC = \ACC(N)$}
		\item{$\REJ = \REJ(N)$}
		\item{$T_{write} = T_{write}(M)~\dot\cup~T_{write}(N)$}
		\item{$T_{direction} = T_{direction}(M)~\dot\cup~T_{direction}(M)$}
		\item{$T_{state} (a,s) =
			\begin{cases}
				T_{state}(N) (a,s) &\text{, if}~s \in S(N) \\
				\INI(N) &\text{, if}~s \in S(M)~\text{and}~T_{state}(M)(a,s) = \ACC(M) \\
				\REJ(N) &\text{, if}~s \in S(M)~\text{and}~T_{state}(M)(a,s) = \REJ(M) \\
				T_{state}(M) (a,s) &\text{, if}~s \in S(M) \\&\text{and}~T_{state}(M)(a,s) \notin \{\ACC(M),\REJ(M)\}
			\end{cases}
		$}
	\end{itemize}
\end{Definition}
\begin{Remark}
	If we set $\INI(N) = \ACC(M)$ and $\REJ(N) = \REJ(M)$ we have $T^\infty(N \circ M) = T^\infty(N) \circ T^\infty(M)$ on $C(M)$, hence the notation. A visualization of this concept can be seen in Figure \ref{turing_machines:basic_notions:fig_concatenation}.
\end{Remark}
\begin{figure}[h]
	\centering
	\begin{tikzpicture}[shorten >=1pt,on grid,auto]
	\node[] (M) {\textbf{M}};
	\node[] (N) [right=8 of M] {\textbf{N}};

	\node[] (Ma) [right = 1.5 of M] {};
	\node[] (Na) [left=1.5 of N] {};
	\node[] (c) [right=4 of M] {};

	\node[state] (M1) [below=1 of Ma] {};
	\node[state] (M0) [left=3 of M1] {\INI};
	\node[state] (MA) [below=3 of M1] {\ACC};
	\node[state] (MR) [below=3 of M0] {\REJ};

	\node[state] (N0) [below=1 of Na] {\INI};
	\node[state] (N1) [right=3 of N0] {};
	\node[state] (NA) [below=3 of N1] {\ACC};
	\node[state] (NR) [below=3 of N0] {\REJ};

	\node[] (tc) [right=4 of M] {};

	\node[] (NoM) [below=6 of c] {\textbf{N $\circ$ M}};

	\node[state] (CM1) [below=6 of M1] {};
	\node[state] (CM0) [below=6 of M0] {\INI};

	\node[state] (CN0) [below=6 of N0] {};
	\node[state] (CN1) [below=6 of N1] {};
	\node[state] (CA) [below=3 of CN0] {\ACC};
	\node[state] (CR) [below=3 of CM1] {\REJ};

	\path[->, color=red]
		(M0) edge (MR)
		(M1) edge (MR)
		(N0) edge (NR)

		(CM0) edge (CR)
		(CM1) edge (CR)
		(CN0) edge (CR)
	;
	
	\path[->, color=green]
		(M1) edge (MA)
		(N0) edge (NA)
		(N1) edge (NA)

		(CM1) edge (CN0)
		(CN0) edge (CA)
		(CN1) edge (CA)
	;

	\path[->]
		(M0) edge [bend right] (M1)
		(M1) edge [bend right] (M0)
		(N0) edge (N1)

		(CM0) edge [bend right] (CM1)
		(CM1) edge [bend right] (CM0)
		(CN0) edge (CN1)
	;

\end{tikzpicture}

	\caption{Example for the concatenation of Turing machines, ignoring the tapes}
	\label{turing_machines:basic_notions:fig_concatenation}
\end{figure}


Clearly, if $M$ and $N$ are both readonly, then so is $N \circ M$. But the same can be said for foolproofness:

\begin{Lemma} \label{turing_machines:lemma_foolproof_concatenation:lemma}
	Let $M$ and $N$ be foolproof Turing machines such that $N \circ M$ is defined. Then $N \circ M$ is foolproof as well.
\end{Lemma}
\proof
Note that
\begin{align*}
	&C(N \circ M) \\
	=~&(A^\Z)^n \times S(N \circ M) \\
	=~&(A^\Z)^n \times (S(N)~\dot\cup~S(M)) \\
	=~&((A^\Z)^n \times S(N))~\dot\cup~((A^\Z)^n \times S(M)) \\
	=~&C(N)~\dot\cup~C(M)
\end{align*}
and thus
\begin{align*}
	&FC(N \circ M) \\
	=~&FC(N \circ M) \cap C(N \circ M) \\
	=~&FC(N \circ M) \cap (C(N)~\dot\cup~C(M)) \\
	=~&(FC(N \circ M) \cap C(N))~\dot\cup~(FC(N \circ M) \cap C(M)) \\
	=~&FC(N)~\dot\cup~FC(M)
\end{align*}
Let $c \in FC(N \circ M)$. As we have just seen, that means $c$ is either in $FC(N)$ or in $FC(M)$.
\paragraph{Case 1: $c \in FC(N)$}
We have
$T(N \circ M)^k (c) = T(N)^k (c) ~\text{for every}~ k \in \N \cup \{\infty\}$
and therefore $T(N \circ M)_{state}^\infty(c) = T(N)_{state}^\infty(c) \in \STOPS$
since $N$ is foolproof.
\paragraph{Case 2: $c \in FC(M)$}
As $M$ is foolproof we know there is some $k \in \N$ such that $T(M)_{state}^k(c) \in \{\ACC(M),\REJ(M)\}$.
In case it is \REJ, we are done.
Otherwise $T(N \circ M)^k(c) = (T(M)_{tape}^k,\INI(N)) =: c^\prime \in FC(N)$.

Thus for $l \geq k$ we have
\begin{align*}
	&T(N \circ M)^l(c) \\
	=~&T(N \circ M)^{l-k}(c^\prime) \\
	=~&T(N)^{l-k}(c^\prime)
\end{align*}
and thus $N \circ M$ stops for $c$ if and only if $N$ stops for $c^\prime$ which is the case as $c^\prime \in FC(N)$ and $N$ is foolproof.
\endproof


\begin{Lemma} \label{turing_machines:lemma_foolproofness:lemma}
	Let $M$ be a readonly Turing machine satisfying
	\begin{align*}
		T_{direction} (\_,s)_i = 1~\text{for every}~ s \in S \setminus \STOPS
	\end{align*}
	or
	\begin{align*}
		T_{direction} (\_,s)_i = -1~\text{for every}~ s \in S \setminus \STOPS
	\end{align*}
	for some $i \in \{1,...,n\}$.
	Then $M$ is foolproof.
\end{Lemma}
\proof
We will only show the case that all directions are positive. The negative case can be proven analogously.
Wlog $i=1$ and even $n=1$. This allows us to leave out the index $i$ wherever it occurs.

Let $c \in FC$. Because of finiteness there is some $k > 0$ such that \IM{a_k = \EMP}.
Thus, since $M$ is readonly we have
\begin{align*}
	&\sum_{j=1}^{k+1} T_{direction}^j (c) < k+1 \\
	\Rightarrow~&\exists~0 < l \leq k: T_{direction}^l (c) \neq 1 \\
	\Rightarrow~&T_{state}^{l-1} (c) \in \STOPS \\
	\Rightarrow~&M ~\text{stops for}~ c
\end{align*}
\endproof


\subsection{Turing machines are basically nice}

To conclude the first chapter we prove a theorem about Turing machines which allows us to use certain classes of Turing machines interchangeably, a feature which becomes important later on.
\begin{Theorem}
	\label{turing_machines:main_theorem:theorem}
	For every Turing machine $M$ on one tape (i.e. $n(M) = 1$), there is a foolproof readonly Turing machine $N$ on two tapes such that
	there is a computable\footnotemark map $f: IC(M) \to C(N)$
	satisfying \IM{f(AC(M)) = AC(N)}.
	\footnotetext{The notion of computability will be discussed in chapter \ref{halting_problem:computability}.}
	\footnotemark
\end{Theorem}
\footnotetext{For a thorough understanding of this notion of similarity between classes of Turing machines see \cite{rog96}.}
\begin{Remark}
	The theorem remains true if we omit the restriction that $M$ has to be on one tape. To see this we must realize that we have a similarity between Turing machines on one tape and Turing machines on any number of tapes just as in the theorem. This is proven e.g. in \cite{sip06}.
\end{Remark}
\proof
	Let $M$ be a Turing machine on one tape.
	We first construct the map $f$ to get an idea what $N$ is supposed to do and then construct $N$ itself.
	But to define $f$ we have to know what $C(N)$ looks like.
	The state-part of $f(c)$ should always be \INI. Therefore it suffices for now to define $A(N)$:
	\begin{align*}
		A(N) := (A(M) \times \{0,1\} \times \{0,1\})~\dot\cup~S(M)~\dot\cup~\{|\}~\dot\cup~\{\EMP\}
	\end{align*}
	Think of a symbol $a \in A(N)$ as either a state or a symbol of $M$, or the ``separator'' symbol $|$.
	If $a$ happens to be a symbol of $M$ it might or might not be ``marked'' in two different ways.
	The first marking should be interpreted as ``position $0$'', the second marking means ``new position''
	and for $a \in A(M)$ we will write
	$\underline{a}$, $\boldsymbol{a}$ and $\underline{\boldsymbol{a}}$ for position $0$-$a$, new position $a$ and new position $0$-$a$ (i.e. $(a,1,0)$, $(a,0,1)$ and $(a,1,1)$) respectively.

	Thus a finite configuration $c \in FC(M)$ may be interpreted as a word over this alphabet:
	The first symbol is the state of $c$. The following symbols are the (mostly) non-\EMP-part of the tape with the symbol at index $0$ underlined.

	Similarly a word over $A(N)$ might denote a sequence of configurations by containing each of those configurations separated by $|$.

	Now we define the tape-part of $f(c)$ to be the word
	\begin{align*}
		c~|~T(c)~|~T^2(c)~|~...~|~T^k(c)
	\end{align*}
	if $T^k(c) = T^{k+1}(c)$ and the respective infinitely long word otherwise (note that every single configuration is still finite).

	There is some ambiguity in this definition as it is not clear exactly how many \EMP-symbols should be in a given configuration.
	This we resolve thus: $c$ itself should be as short as possible, that is it contains the underlined symbol, all non-\EMP-symbols and any \EMP~in between, but no more.
	Two consecutive configurations should have the same length if possible. If not, the second configuration may be one symbol longer.
	This occurs if the underlined symbol is rightmost (or leftmost) and $T_{direction}$ is $1$ (or $-1$).
	In this and only this case, the underlined symbol in the second configuration should be bold (i.e. marked as new position).

	Now we are ready to define the Turing machine $N$ itself. This we do in four stages as outlined below:
	\begin{figure}[H]
		\centering
		\begin{tikzpicture}[shorten >=1pt,on grid,auto]
	\node[] (0) {Turing machine};
	\node[] (1) [below=2 of 0] {foolproof readonly Turing machine};
	\node[] (2) [below=2 of 1] {Turing dynamical system};
	\node[] (5) [below=2 of 2] {$(\Z_2 \wr \Z)^2$-CW-Complex};
	
	\node[] (10) [right=8 of 0] {accepts some input};
	\node[] (11) [right=8 of 1] {accepts some input};
	\node[] (12) [right=8 of 2] {accepts more than a null set};
	\node[] (15) [right=8 of 5] {trivial \ltwo-betti-number};

	\path[->]
		(0) edge (1)
		(1) edge (2)
		(2) edge (5)
	;
\end{tikzpicture}

	%	\caption{Stage 1 - checking input is of correct form}
	%	\label{turing_machines:main_theorem:fig_stage1}
	\end{figure}
	
	In detail, the stages work like this:
	The first one checks if
	\begin{itemize}
		\item { Both tapes are equal (in an area bordered by \EMP~around $0$). }
		\item { The tapes contain a sequence of words, each terminated by $|$. }
		\item { Each of those words is a configuration. That is it starts with a state and every other symbol is from the alphabet of $M$, with exactly one symbol underlined. }
		\item { if a symbol is bold it is also underlined and either the first or last symbol of the tape part of a configuration }
	\end{itemize}
	It does not yet check that the lengths of two consecutive configurations differ by at most one symbol. This is deferred to the fourth stage.

	The set of states of this first stage is $\{\INI,\ACC,\REJ,1,2,3,4,4^\prime\}$.
	$T_{direction}(c)$ is 1 for both tapes, unless $T_{state}(c) \in \STOPS$, in which case it is 0. $T_{write}$ is determined by the readonly-property.
	
	\begin{figure}
		\centering
		\begin{tikzpicture}[shorten >=1pt,on grid,auto]
	\node[state] (0) {\INI};
	\node[state] (1) [below=6 of 0] {2};
	\node[state] (2) [right=6 of 1] {3};
	\node[state] (3) [below=6 of 2] {4};
	\node[state] (4) [left=6 of 3] {1};
	\node[state] (6) [below=6 of 4] {\ACC};
	\node[state] (5) [right=12 of 6] {\textbf{4}};

	\path[->]
		(0) edge [ left ]
			node {$\begin{matrix}s \\ s\end{matrix}, s \in S(M)$ }
			(1)
		(1) edge [ above ]
			node {$\begin{matrix}x \\ x\end{matrix}, x \in A(M)$ }
			(2)
		(2) edge [ loop , in=15 , out=75 , distance=50 , above ]
			node {$\begin{matrix}x \\ x\end{matrix}, x \in A(M)$}
			(2)
		(2) edge [ right ]
			node {$\begin{matrix}\underline{x} \\ \underline{x}\end{matrix}, x \in A(M)$}
			(3)
		(3) edge [ loop , in=285 , out=345 , distance=50 , below ]
			node {$\begin{matrix}x \\ x\end{matrix}, x \in A(M)$}
			(3)
		(3) edge [ below ]
			node {$\begin{matrix}| \\ |\end{matrix}$}
			(4)
		(4) edge [ left ]
			node {$\begin{matrix}\EMP \\ \EMP\end{matrix}$}
			(6)
		(4) edge [ left ]
			node {$\begin{matrix}s \\ s\end{matrix}, s \in S(M)$}
			(1)
		(2) edge [ bend left ]
			node {$\begin{matrix}\underline{\boldsymbol{x}} \\ \underline{\boldsymbol{x}} \end{matrix}, x \in A(M)$}
			(5)
		(5) edge [ bend left ]
			node {$\begin{matrix}| \\ |\end{matrix}$}
			(4)
		(1) edge [ bend left ]
			node {$\begin{matrix}\underline{x} \\ \underline{x}\end{matrix}, x \in A(M)$}
			(3)
		(1) edge [ bend right ]
			node {$\begin{matrix}\underline{\boldsymbol{x}} \\ \underline{\boldsymbol{x}}\end{matrix}, x \in A(M)$}
			(3)
	;
\end{tikzpicture}

		\caption{Stage 1 - checking input is of correct form}
		\label{turing_machines:main_theorem:fig_stage1}
	\end{figure}

	$T_{state}$ can be seen in Figure \ref{turing_machines:main_theorem:fig_stage1}:
	For some configuration $s \in S, a_1,a_2 \in A$, we set \IM{T_{state}((a_1,a_2),s) := t} if there is an edge from $s$ to $t$ whos label matches $(a_1,a_2)$.
	If there is no such edge for any $t$, then \IM{T_{state}((a_1,a_2),s) = \REJ}.

	To understand what is happening here, let us give more meaningful names to the states:
	\begin{align*}
		1 &\mapsto~\text{start of a new configuration} \\
		2 &\mapsto~\text{start of the tape} \\
		3 &\mapsto~\text{somewhere in the non-positive part of the tape} \\
		4 &\mapsto~\text{somewhere in the positive part of the tape} \\
		4^\prime &\mapsto~\text{previous symbol was bold and not the first one of the tape} \\
	\end{align*}
	The loop \IM{1 \to 2 \to 3 \to 4 \to 1} only goes through if both tapes are equal and their $0$-position is the start of a configuration as described above.
	Looping once will advance both tapes by exactly one configuration.
	$4^\prime$ is required to make sure that at most the first and last symbol are bold.

	The second stage shifts both tapes by $-1$ until they hit an \EMP-symbol again, that is they are one symbol left of their original position.
	The third stage shifts both tapes by $1$ again.
	Thus the concatenation of the first three stages fulfills \IM{T^\infty(a,\INI) = (a,\ACC)} if $a$ is of correct form and is undefined or has a different state otherwise.
	Note that the first stage does not stop for $f(c)$, if $M$ does not stop for $c$. But this is not a problem as in particular it does not accept $c$.

	Now that we know the input is the same on both tapes and of the form
	\begin{align*}
		c_0~|~c_1~|~...~|~c_k
	\end{align*}
	for some $c_i \in C(M)$ it remains to check that \IM{c_i = T(M)^i(c_0)} for all \IM{i \in \{1,...,k\}}.

	This is done by the fourth stage.
	Its set of states is
	\begin{align*}
		(\{-1,0,1,2,3,4,5\} \times A(M) \times S(M))~\cup~\{\INI,\ACC,\REJ\}
	\end{align*}
	This stage should be readonly again so $T_{write}$ is already defined.

	$T_{state}$ and $T_{direction}$ may be gathered from Figure \ref{turing_machines:main_theorem:fig_stage4}.

	\begin{figure}
		\centering
		\begin{tikzpicture}[shorten >=1pt,on grid,auto]
	\node[state] (0) {\INI};
	\node[state] (1) [below=4 of 0] {-1};
	\node[state] (2) [below=4 of 1] {0};
	\node[state] (3) [below=4 of 2] {1};
	\node[state] (4) [right=8.5 of 3] {2};
	\node[state] (5) [above=7 of 4] {3};
	\node[state] (6) [right=5.5 of 4] {4};
	\node[state] (7) [below=7 of 4] {5};
	\node[state] (8) [below=7 of 3] {\ACC};

	\path[->]
		(0) edge [ left ]
			node {$\begin{matrix}s \\ s\end{matrix} \mapsto \begin{matrix}0 \\ 1\end{matrix}$ }
			(1)
		(1) edge [ loop left ]
			node {$\begin{matrix}* \\ *\end{matrix} \mapsto \begin{matrix}0 \\ 1\end{matrix}$}
			(1)
		(1) edge [ left ]
			node {$\begin{matrix}* \\ \underline{a}\end{matrix} \mapsto \begin{matrix}0 \\ 1\end{matrix},(a,*)$}
			(2)
		(2) edge [ loop left ]
			node {$\begin{matrix}* \\ *\end{matrix} \mapsto \begin{matrix}0 \\ 1\end{matrix}$}
			(2)
		(2) edge [ left ]
			node {$\begin{matrix}* \\ |\end{matrix} \mapsto \begin{matrix}0 \\ 1\end{matrix}$}
			(3)
		(3) edge [ left ]
			node {$\begin{matrix}\ACC \\ *\end{matrix}$}
			(8)
		(3) edge [ bend left , blue ]
			node {$(a,*),\begin{matrix}s \\ T_{state}(a,s)\end{matrix} \mapsto \begin{matrix}1 \\ 1\end{matrix},(a,s)$}
			(4)
		(4) edge [ bend left , green ]
			node {$\begin{matrix}| \\ |\end{matrix} \mapsto \begin{matrix}1 \\ 1\end{matrix}$}
			(3)
		(4) edge [ loop right ]
			node {$\begin{matrix}x \\ x\end{matrix} \mapsto \begin{matrix}1 \\ 1\end{matrix}$}
			(4)
		(4) edge [ loop left , red ]
			node {$\begin{matrix}
				(a,s),\begin{matrix}\underline{a} \\ \underline{T_{write}(a,s)}\end{matrix} \mapsto \begin{matrix}1 \\ 1\end{matrix},(T_{write}(a,s),s) \\
				\text{if}~T_{direction}(a,s) = 0
			\end{matrix}$}
			(4)
		(4) edge [ bend left , near end , red ]
			node {$\begin{matrix}
				(a,s),\begin{matrix}x \\ \underline{x}\end{matrix} \mapsto \begin{matrix}1 \\ 1\end{matrix},(x,s) \\
				\text{if}~T_{direction}(a,s) = -1
			\end{matrix}$}
			(5)
		(5) edge [ bend left , near start ]
			node {$(x,s),\begin{matrix}\underline{a} \\ T_{write}(a,s)\end{matrix} \mapsto \begin{matrix}1 \\ 1\end{matrix},(x,s)$}
			(4)
		(4) edge [ bend left , red ]
			node {$\begin{matrix}
				(a,s),\begin{matrix}\underline{a} \\ \underline{\boldsymbol{x}}\end{matrix} \mapsto \begin{matrix}0 \\ 1\end{matrix},(x,*) \\
				\text{if}~T_{direction}(a,s) = -1
			\end{matrix}$}
			(6)
		(6) edge [ bend left ]
			node {$(x,s),\begin{matrix}\underline{a} \\ T_{write}(a,s)\end{matrix} \mapsto \begin{matrix}1 \\ 1\end{matrix},(x,s)$}
			(4)
		(4) edge [ bend left , near end , red ]
			node {$\begin{matrix}
				(a,s),\begin{matrix}\underline{a} \\ T_{write}(a,s)\end{matrix} \mapsto \begin{matrix}1 \\ 1 \end{matrix}, (a,s) \\
				\text{if}~T_{direction}(a,s) = 1
			\end{matrix}$}
			(7)
		(7) edge [ bend left=10 , near start ]
			node {$\begin{matrix}x \\ \underline{x}\end{matrix} \mapsto \begin{matrix}1 \\ 1\end{matrix}$}
			(4)
		(7) edge [ bend left ]
			node {$\begin{matrix}| \\ \underline{\boldsymbol{x}}\end{matrix} \mapsto \begin{matrix}0 \\ 1\end{matrix}$}
			(4)
	;
\end{tikzpicture}

		\caption{Stage 4 - checking configurations are consecutive in a $T$-applying sense}
		\label{turing_machines:main_theorem:fig_stage4}
	\end{figure}

	This requires some explanation. First of all, think of the $A(M)$ and $S(M)$ parts of the states as ``memory'' as this is more descriptive of its function.
	An edge \IM{p \to q} with label
	\begin{align*}
		(a,s),\begin{matrix} x \\ y \end{matrix} \mapsto \begin{matrix} i \\ j \end{matrix}, (b,t)
	\end{align*}
	should be interpreted as
	\begin{align*}
		T_{state}((x,y),(p,a,s)) &:= (q,b,t) \\
		T_{direction}((x,y),(p,a,s)) &:= (i,j)
	\end{align*}
	$(a,s)$ may be omitted or any of $a$, $s$, $x$ and $y$ may be $*$. This means that the above definition may be applied, whatever the respective value is.
	If $(a,s)$ and $(b,t)$ are both omitted, assume \IM{(b,t) = (a,s)} and $a$ and $s$ may be any value again.
	If $b$ or $t$ is $*$, assume \IM{b = \EMP}~and \IM{t = \INI}~respectively for sake of disambiguity but consider the concrete value not to be actually important.

	Now that we understand how to read the diagram we can try to gather what is happening. To this end we again give more descriptive names to the states:
	\begin{align*}
		-1 &\mapsto~\text{advance second tape one state, negative part} \\
		0 &\mapsto~\text{advance second tape one state, positive part} \\
		1 &\mapsto~\text{start of configuration} \\
		2 &\mapsto~\text{somewhere in configuration} \\
		3 &\mapsto~\text{direction = -1} \\
		4 &\mapsto~\text{direction = -1, new position} \\
		5 &\mapsto~\text{direction = 1}
	\end{align*}
	Initially both tapes should be at their leftmost position.
	The states \INI, -1 and 0 advance the second tape by one configuration.
	State 1 expects the first tape to be positioned at the left end of some configuration $c_i$ and the second tape at configuration $c_{i+1}$.
	The ``memory'' should contain the underlined symbol of $c_i$, i.e. $T_{tape}^0(c_i)_0$.
	Then follows a sequence of cycles of the form
	\begin{align*}
		1 \to 2 \to 2 \to ... \to 2 \to (2~\text{or}~3~\text{or}~4~\text{or}~5) \to 2 \to ... \to 2 \to 1
	\end{align*}
	Every such cycle shifts both tapes by one configuration while checking that the second one is the image of the first one under $T$.
	The first edge \IM{1 \to 2} checks the correct application of $T_{state}$ and puts $T_{state}^0(c_i)$ into memory.
	Thus we may determine $T_{write}(c_i)$ and $T_{direction}(c_i)$ from memory.
	The correct application of these maps is checked when the underlined symbol of either tape is encountered and done by a case distinction that involves states $3$ through $5$ and a loop on state $2$.

	Note that both tapes are always shifted the same way, except if the second tape contains a bold symbol, in which case it is shifted by one more (for every such symbol, which by assumption is at most one).
	Thus the correct length of the configurations is checked.

	If this way we reach a configuration with state \ACC, we have confirmed the existence of a sequence of consecutive configurations
	\begin{align*}
		(c, T(c), T^2(c) ... T^k(c) = T^\infty(c) = (*,\ACC))
	\end{align*}
	and thus we know $M$ accepts $c$.

	Now define $N$ to be the concatenation of all four stages. Let $c \in IC(N)$.
	If $c \notin f(IC(M))$, either the first or fourth stage rejects it.
	If $c = f(c^\prime)$ for some $c^\prime \in IC(M)$, the first stage accepts it and the fourth stage accepts $c$ if and only if $M$ accepts $c^\prime$. Thus \IM{f(AC(M)) = AC(N)}.
%
	Clearly, $N$ is readonly and every stage has that $(T_{direction})_2$ is constant and non-zero (unless the next state is \ACC~or \REJ)
	and therefore is foolproof by Lemma \ref{turing_machines:lemma_foolproofness:lemma}
	and so is $N$ by Lemma \ref{turing_machines:lemma_foolproof_concatenation:lemma}.

\endproof


%\begin{Definition}
%	If $\mathcal{M}$ and $\mathcal{N}$ are classes of Turing machines, $\mathcal{N}$ has \emph{no lesser computability} than $\mathcal{M}$ (write $\mathcal{M} \leq \mathcal{N}$),
%	if for every $M \in \mathcal{M}$ there is an $N \in \mathcal{N}$ and a computable map $F: IC(M) \to C(N)$ such that \IM{F(AC(M) = AC(N)}.
%\end{Definition}

%Thm same computability

%\begin{Theorem}\label{thm_computability}
%	The following classes have the same computability:
%
%	\begin{enumerate}
%		\item{Turing machines on 1 tape}
%			\label{thm_computability:tm1}
%		\item{readonly Turing machines on 2 tapes}
%			\label{thm_computability:r2}
%		\item{readonly Turing machines on any number of tapes}
%			\label{thm_computability:r}
%		\item{foolproof readonly Turing machines on 2 tapes}
%			\label{thm_computability:fr2}
%	\end{enumerate}
%\end{Theorem}
%
%Later we will only actually need
%\mbox{\ref{thm_computability:tm1} $\rightsquigarrow$ \ref{thm_computability:fr2}}
%but \ref{thm_computability:r2} is a useful step in the proof,
%\ref{thm_computability:r} we basically get for free
%and \mbox{\ref{thm_computability:fr2} $\rightsquigarrow$ \ref{thm_computability:tm1}} 
%shall be proven for sake of completeness.
%
%\proof
%\mbox{\ref{thm_computability:r2} $\rightsquigarrow$ \ref{thm_computability:r}} is clear ($N = M$ and $F = id$).
%We will prove
%\mbox{\ref{thm_computability:tm1} $\rightsquigarrow$ \ref{thm_computability:r2}},
%\mbox{\ref{thm_computability:r} $\rightsquigarrow$ \ref{thm_computability:fr2}} and
%\mbox{\ref{thm_computability:fr2} $\rightsquigarrow$ \ref{thm_computability:tm1}}
%each by first providing the map $F$ which should preserve acceptance and then the Turing machine $N$, making sure $N$ rejects every \IM{c \in IC(N) \setminus F(IC(M))}.
%
%\subsubsection*{\ref{thm_computability:tm1} $\rightsquigarrow$ \ref{thm_computability:r2}}
%
%Let $M$ be a Turing machine on 1 tape. Consider a tape over the alphabet \IM{A(M) \cup S(M)}. This tape may contain a finite configuration $(a_{-l},\cdots,a_k,s) \in FC(M)$ by putting $s$ leftmost, followed by the $a_i$ and marking $a_0$ and no other symbol. The marking is important so we can tell the position of the $a_i$ on the tape.
%If we add to the alphabet some separator symbol $|$ a tape can contain a whole sequence of configurations. We shall now try to create a readonly Turing machine on 2 tapes $N$ that checks if such a sequence is of the form 
%\begin{align*}
%	(c_0, T(c_0), T^2(c_0), ... T^m(c_0))
%\end{align*}
%where \IM{T^m(c_0) = T^{m+1}(c_0)} and \IM{T = T^M}.
%
%In order to do this we should first check that both tapes contain the same input. At the same time we can check whether the input is of the form
%\begin{align*}
%	|c|c^\prime|c^{\prime\prime}|\cdots
%\end{align*}
%for some finite configurations \IM{c,c^\prime,c^{\prime\prime}} etc
%and reject if any of these conditions are not met.
%
%Checking if some word bounded by separators actually describes a configuration can be done by checking if the first symbol is in $S$, every other symbol is in $A$ and when encountering a marking memorizing the fact and rejecting if another marking is met.
%
%Afterwards we move both tapes to their original positions and advance the second tape by one configuration.
%
%Now we only need to check that if the first tape is at some configuration $\tilde{c}$, the second tape is at the start of $T^M(\tilde{c})$, advancing both tapes by one configuration in the process and repeat until either this is not true, in which case we reject, or we reach the end of the second tape, in which case we accept if an only if the configurations we considered last are identical and have state $ACCEPT$.
%
%To achieve this let $(a,s)$ and $(b,t)$ be the configurations on the first and second tape respectively. We first read the states $s$ and $t$ and memorize them.
%Then we have both tapes skip any initial unmarked $EMPTY$ symbols, as those may or may not be specified explicitly without affecting equality of configurations.
%Then both tapes are advanced one step at a time until we find a marking on either tape. Up to this point both tapes should be equal or we reject.
%Now things depend on whether the marking was found on the first or second tape or both.
%In case there was a marking on the first tape and this tape contains symbol $a$ at position $0$, the second tape should contain the symbol \IM{T^M_A(a,s)} and be marked if and only if \IM{T^M_D(a,s) = 0}. Furthermore we should have that \IM{T_S(a,s) = t}. All of these conditions can easily be checked as all the relevant information is either in memory or on the observable parts of the tapes.
%If any of these conditions is not met we again reject.
%If \IM{T^M_D(a,s) = -1} we should already have encountered a marking on the second tape. But we have not, therefore we reject.
%
%If there was a marking on the first tape and \IM{T^M_D(a,s) = 1}, or there was no marking on the first tape but on the second one, the next position of the respective other tape should be marked, a fact we memorize. Then advance both tapes one step and check if the markings are correct according to what we memorized. If now there is a marking on the first tape perform the above check concerning the symbol on the second tape.
%
%Then we continue advancing both tapes expecting no differences barring some $EMPTY$ symbols at the end.
%This way we reach the $|$ marking the end of the configuration on both tapes if and only if \IM{T^M(a,s) = (b,t)}
%
%During this whole process if we ever encounter a difference in the configurations (ignoring any unmarked $EMPTY$ at the starts and ends) we should remember the fact so that we can decide whether or not to accept if this configuration happened to be the last one on the second tape.
%
%\subsubsection*{\ref{thm_computability:r} $\rightsquigarrow$ \ref{thm_computability:fr2}}
%
%Let $M$ be a read only Turing machine. We first restrict to the case that $M$ is on just one single tape.
%The idea here is as follows: If we somehow knew that \IM{T_D(\_,s) = 1} for every \IM{s \in S \setminus \{ACCEPT,REJECT\}} foolproofness would easily follow since unless the machine has already stopped, the tape is advanced by 1 every step and a readonly Turing machine may do so only as long as it does not hit any $EMPTY$ symbol because of our further requirement on such machines.
%For any finite configuration this will happen after finitely many steps and thus the Turing machine would stop.
%
%To construct such a Turing machine we consider the ``reading word'' $r$ of any input. This word consists of the entries on the tape as they are consecutively read by $M$: If $M$ stops for an initial configuration $c$ after $K \in \N$ steps, we set
%\begin{align*}
%	r^M(c)_i := (T^M_A)^i(c)~\forall k \in \{0,...,k\}
%\end{align*}
%Now we can easily construct a foolproof Turing machine $N$ that simulates the behaviour of $M$ when applied to the reading word of any input to $M$ by setting
%\begin{align*}
%	T^N_D(\_,s) := \begin{cases}0&\text{, if}~s \in \{ACCEPT,REJECT\} \\1&\text{, otherwise}\end{cases}
%\end{align*}
%and
%\begin{align*}
%	T^N_S := T^M_S
%\end{align*}
%Now we only need to make sure that a given input actually is a reading word. Here we require our second tape. The input to the second tape should be the original input to $M$. Let that input be called $c$. Then the input to the first tape should be the reading word $r^M(c)$, with each position $i$ on the tape marked by $(T^M_D)^{i+1}(c)$.
%
%We divide $N$ into three phases: The first phase checks whether the input to the first tape is actually the reading word of the input to the second tape when following directions as indicated by the markings, which should correspond to the movement of the tape in $M$.
%That is if $a$ and $b$ denote the tapes, $m \in \{-1,0,1\}^\Z$ denotes the markings on the first tape and $s \in S$ we have
%\begin{align*}
%	T^N_S(((a,m),b),s) = \begin{cases}T^M_S(b,s)&\text{, if}~a_0 = b_0~\text{and}~m_0 = T^M_D(b,s)\\REJECT&\text{, otherwise}\end{cases}
%\end{align*}
%and further in case we did not reject
%\begin{align*}
%	T^N_D(((a,m),b),s) = \begin{pmatrix}1\\m_0\end{pmatrix}
%\end{align*}
%The second phase resets the tapes to their initial positions, that is moves each of them left until they hit an $EMPTY$ symbol after which the tapes are moved right by one step.
%
%Finally the third phase simulates $M$.
%
%This is still foolproof, as each of the three phases, when regarded as a single Turing machine is foolproof and the concatenation of foolproof Turing machines is clearly foolproof again since when one Turing machine eventually stops after finitely many steps, the second machine is started with some configuration, which again stops after finitely many steps and so on.
%
%For the case where $M$ has multiple tapes we first realize that we can combine mutliple tapes into one by replacing the alphabet $A$ by $A^n$. Doing so loses us the ability to move the tapes independently. But this is not much of a problem since still the second tape may contain the combined inputs to $M$ and the first tape holds their combined reading words, each with appropriate directions.
%We check, one after another, if the reading words are correct and then simulate $M$. This is possible since for the simulation we only need to consider the first tape, which contains the consecutive symbols on all the tapes as observed by $M$.
