\begin{Definition}
	A Turing machine is said to
	\begin{itemize}
		\item{\emph{accept $c$}, if $T_{state}^\infty(c)$ is defined and equals \ACC}
		\item{\emph{reject $c$}, if $T_{state}^\infty(c)$ is defined and equals \REJ}
		\item{\emph{stop for $c$}, if it accepts $c$ or rejects $c$}
		\item{\emph{stop}, if it stops for any initial configuration}
		\item{be \emph{foolproof}, if it stops for any finite configuration}
		\item{be \emph{readonly}, if $T_{write}(a,s) = a ~\text{for every}~ a \in A^n, s \in S$}
	\end{itemize}
	Of a readonly Turing machine we will further require that it does not ``cross'' the \EMP-symbol, that is for every $N \in \N$ and $(a,s) \in C$ the following holds:
	\begin{align*}
		a_{l,i} \neq \EMP~\forall~l \in \{0,...,(\sum_{j=1}^N T_{direction}^j(a,s)_i) -1\}
	\end{align*}
	if the sum is positive and analogous if the sum is negative.\footnotemark

	Here $T_{direction}^j$ denotes the direction of shift in the $j$'th iteration of $T$.

	\footnotetext{Using the notion of similarity from Theorem \ref{turing_machines:main_theorem:theorem} we can see that this is not a real restriction.}
\end{Definition}

The subsets of $IC$ that $M$ accepts, rejects and stops for will be denoted by $AC$, $RC$ and $SC$ respectively.

Now let us present a method of turning multiple Turing machines into one by ``first executing one, then the other one''. This will be useful later as it gives us a method to divide a large Turing machine into more managable parts.

\begin{Definition}
	Let $M$ and $N$ be Turing machines such that $A(M) = A(N)$ and $n(M) = n(N)$, that is, their configurations differ only by the states. Then we define their \emph{concatenation} $N \circ M$ as the Turing machine defined by the following data:
	\begin{itemize}
		\item{$n := n(M) = n(N)$}
		\item{$A := A(M) = A(N)$}
		\item{$\EMP := \EMP(M) = \EMP(N)$}
		\item{$S := S(M)~\dot\cup~S(N)$}
		\item{$\INI := \INI(M)$}
		\item{$\ACC := \ACC(N)$}
		\item{$\REJ := \REJ(N)$}
		\item{$T_{write} := T_{write}(M)~\dot\cup~T_{write}(N)$}
		\item{$T_{direction} := T_{direction}(M)~\dot\cup~T_{direction}(N)$}
		\item{$T_{state} (a,s) :=
			\begin{cases}
				T_{state}(N) (a,s) &\text{, if}~s \in S(N) \\
				\INI(N) &\text{, if}~s \in S(M)~\text{and}~T_{state}(M)(a,s) = \ACC(M) \\
				\REJ(N) &\text{, if}~s \in S(M)~\text{and}~T_{state}(M)(a,s) = \REJ(M) \\
				T_{state}(M) (a,s) &\text{, if}~s \in S(M) \\&\text{and}~T_{state}(M)(a,s) \notin \{\ACC(M),\REJ(M)\}
			\end{cases}
		$}
	\end{itemize}
\end{Definition}
\begin{Remark}
	If we set $\INI(N) = \ACC(M)$ and $\REJ(N) = \REJ(M)$ we have $T^\infty(N \circ M) = T^\infty(N) \circ T^\infty(M)$ on $C(M)$, hence the notation. A visualization of this concept can be seen in Figure \ref{turing_machines:basic_notions:fig_concatenation}.
\end{Remark}
\begin{figure}[h]
	\centering
	\begin{tikzpicture}[shorten >=1pt,on grid,auto]
	\node[] (M) {\textbf{M}};
	\node[] (N) [right=8 of M] {\textbf{N}};

	\node[] (Ma) [right = 1.5 of M] {};
	\node[] (Na) [left=1.5 of N] {};
	\node[] (c) [right=4 of M] {};

	\node[state] (M1) [below=1 of Ma] {};
	\node[state] (M0) [left=3 of M1] {\INI};
	\node[state] (MA) [below=3 of M1] {\ACC};
	\node[state] (MR) [below=3 of M0] {\REJ};

	\node[state] (N0) [below=1 of Na] {\INI};
	\node[state] (N1) [right=3 of N0] {};
	\node[state] (NA) [below=3 of N1] {\ACC};
	\node[state] (NR) [below=3 of N0] {\REJ};

	\node[] (tc) [right=4 of M] {};

	\node[] (NoM) [below=6 of c] {\textbf{N $\circ$ M}};

	\node[state] (CM1) [below=6 of M1] {};
	\node[state] (CM0) [below=6 of M0] {\INI};

	\node[state] (CN0) [below=6 of N0] {};
	\node[state] (CN1) [below=6 of N1] {};
	\node[state] (CA) [below=3 of CN0] {\ACC};
	\node[state] (CR) [below=3 of CM1] {\REJ};

	\path[->, color=red]
		(M0) edge (MR)
		(M1) edge (MR)
		(N0) edge (NR)

		(CM0) edge (CR)
		(CM1) edge (CR)
		(CN0) edge (CR)
	;
	
	\path[->, color=green]
		(M1) edge (MA)
		(N0) edge (NA)
		(N1) edge (NA)

		(CM1) edge (CN0)
		(CN0) edge (CA)
		(CN1) edge (CA)
	;

	\path[->]
		(M0) edge [bend right] (M1)
		(M1) edge [bend right] (M0)
		(N0) edge (N1)

		(CM0) edge [bend right] (CM1)
		(CM1) edge [bend right] (CM0)
		(CN0) edge (CN1)
	;

\end{tikzpicture}

	\caption{Example for the concatenation of Turing machines, ignoring the tapes}
	\label{turing_machines:basic_notions:fig_concatenation}
\end{figure}
