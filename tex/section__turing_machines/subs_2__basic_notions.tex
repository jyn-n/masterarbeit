A Turing machine is said to
\begin{itemize}
	\item{\emph{accept $c$}, if $T_{state}^\infty(c)$ is defined and equals \ACC}
	\item{\emph{reject $c$}, if $T_{state}^\infty(c)$ is defined and equals \REJ}
	\item{\emph{stop for $c$}, if it accepts $c$ or rejects $c$}
	\item{\emph{stop}, if it stops for any initial configuration}
	\item{be \emph{foolproof}, if it stops for any finite configuration}
	\item{be \emph{readonly}, if $T_{write}(a,s) = a~\forall a \in A^n, s \in S$}
\end{itemize}
Of a readonly Turing machine we will further require that it does not ``cross'' the \EMP-symbol, that is for every $N \in \N$ and $a \in (A^\Z)^n$ the following holds:
\begin{align*}
	a_{l,i} \neq \EMP~\forall~l \in \{0,...,(\sum_{j=1}^N T_{direction}^j(a,s)_i) -1\}
\end{align*}
if the sum is positive and analogous if the sum is negative.\footnotemark

\footnotetext{Using the notion of similarity from Theorem \ref{turing_machines:main_theorem:theorem} we can see that this is not a real restriction.}

The subsets of $IC$ that $M$ accepts, rejects and stops for will be denoted by $AC$, $RC$ and $SC$ respectively.

\remark $M$ stops for $c$ if and only if there is some $k \in \N_0$ such that \IM{T_{state}^k(c) \in \STOPS}

Now let us present a method of turning multiple Turing machines into one by ``first executing one, then the other one''. This will be useful later as it gives us a method to divide a large Turing machine into more managable parts.

\begin{Definition}
	Let $M$ and $N$ be Turing machines such that $A(M) = A(N)$ and $n(M) = n(N)$. Then we define their \emph{concatenation} $N \circ M$ as the Turing machine defined by the following data:
	\begin{itemize}
		\item{$n = n(M) = n(N)$}
		\item{$A = A(M) = A(N)$}
		\item{$\EMP = \EMP(M) = \EMP(N)$}
		\item{$S = S(M)~\dot\cup~S(N)$}
		\item{$\INI = \INI(M)$}
		\item{$\ACC = \ACC(N)$}
		\item{$\REJ = \REJ(N)$}
		\item{$T_{write} = T_{write}(M)~\dot\cup~T_{write}(N)$}
		\item{$T_{direction} = T_{direction}(M)~\dot\cup~T_{direction}(M)$}
		\item{$T_{state} (a,s) =
			\begin{cases}
				T_{state}(N) (a,s) &\text{, if}~s \in S(N) \\
				\INI(N) &\text{, if}~s \in S(M)~\text{and}~T_{state}(M)(a,s) = \ACC(M) \\
				\REJ(N) &\text{, if}~s \in S(M)~\text{and}~T_{state}(M)(a,s) = \REJ(M) \\
				T_{state}(M) (a,s) &\text{, if}~s \in S(M) \\&\text{and}~T_{state}(M)(a,s) \notin \{\ACC(M),\REJ(M)\}
			\end{cases}
		$}
	\end{itemize}
\end{Definition}
\remark If we set $\INI(N) = \ACC(M)$ and $\REJ(N) = \REJ(M)$ we have $T^\infty(N \circ M) = T^\infty(N) \circ T^\infty(M)$ on $C(M)$, hence the notation.
