\section{Turning a Turing machine into a TDS} \label{tm_to_tds}

We will now see that the concept of Turing dynamical systems is a generalization of Turing machines.
We will consider a readonly Turing machine and construct a Turing dynamical system that keeps all of the information from the Turing machine.
A similar transition is possible for Turing machines that are not readonly but readonlyness allows for a much more direct approach and is therefore preferred.

Let $M$ be a readonly Turing machine.
Then we define the Turing dynamical system $M^\prime$ as follows:

Without loss of generality we may assume that \IM{\left\vert A(M) \right\vert = 2^k} for some \IM{k \in \N}.
Otherwise we add random symbols to the alphabet until it has the desired size.
Similarly we assume that \IM{\left\vert S(M) \right\vert = 2^l}.
Now consider the group $\Z_2$.
We set
\begin{align*}
	X := {(\Z_2^{k})^{\Z^{n(M)}}} \times \Z_2^l
\end{align*}
Note that this is basically $C(M)$ with an added group structure. On $Z_2$ we have a probability measure defined by
\begin{align*}
	\mu(0) = \mu(1) = \frac12
\end{align*}
This measure extends to a unique probability measure on X by the Andersen-Jessen theorem. %maybe reference?



