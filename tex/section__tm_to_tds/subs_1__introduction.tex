\subsection{Construction}
\label{tm_to_tds:construction}

Let $M$ be a readonly Turing machine.\footnotemark~
Then we define the Turing dynamical system $M^\prime$ as follows:
\footnotetext{Remark \ref{remarks:tds_construction} gives us an approach for non-readonly Turing machines.}

First of all extend the alphabet of $M$ by the newly introduced $\DELIM$-symbol. This will be required for formal reasons later on. If it is ever encountered on a tape, $M$ should immediately reject.

We will require that \IM{\left\vert A(M)~\cup~\{\DELIM\} \right\vert = 2^k} for some \IM{k \in \N}.
If this is not true, we add random symbols to the alphabet until it has the desired size. By rejecting those symbols whenever they are encountered we bascially do not change the accepting behaviour of $M$.
Similarly we may assume that \IM{\left\vert S(M) \right\vert = 2^l-1}.
Now consider the group $\Z_2$.
We set
\begin{align*}
	X := {{((\Z_2^{k})^\Z)}^{n(M)}} \times \Z_2^l
\end{align*}
Into this we can embed the configurations of $M$ by embedding $A(M)~\cup~\{\DELIM\}$ into $\Z_2^k$ and thereby mapping the tapes to the left factor and the state in any way to the right factor except $0$.

Let us fix any such mapping and regard $C(M)$ as a subset of $X$ from now on.

\begin{Remark}
	For now we could of course have used some $\Z_m$ instead of $\Z_2^k$ to represent the alphabet and there is no obvious reason why we do not do so.
	But later on, namely in Theorem \ref{the_zero_divisor_problem:grabowskis_theorem} we will require that our Turing dynamical systems use groups that are products of $\Z_2$ so we construct them like this right from the beginning.
\end{Remark}

\begin{Notation}
	We introduce a notation for a certain class of subsets of $X$:

	Let $a_{-j},...,a_{m} \in \Z_2^k$ for some $j, m \in \N_0$ and $s_0 \in \Z_2^l$. In case the Turing machine $M$ has just one tape, we set
	\begin{align*}
		\begin{bmatrix}
			a_{-j},...,\underline{a_0},...,a_m
		\end{bmatrix} := \{ (x,s) \in X = (\Z_2^k)^\Z \times \Z_2^l ~|~ x_i = a_i ~\forall i \in \{-j,...,m\}\} 
	\end{align*}
	and
	\begin{align*}
		\begin{bmatrix}
			a_{-j},...,\underline{a_0},...,a_m
		\end{bmatrix} [ s_0 ] := \{ (x,s) \in \begin{bmatrix} a_{-j},...,\underline{a_0},...,a_m \end{bmatrix} ~|~ s = s_0 \}
	\end{align*}
	For Turing machines with multiple tapes we use a similar notation.
\end{Notation}

On $\Z_2$ we have a probability measure defined by
\begin{align*}
	\mu(0) = \mu(1) = \frac12
\end{align*}
This measure extends to the normalized Haar measure\footnotemark~ on $X$.
This measure has the nice property that calculating the measure for a certain class of subsets of $X$ is quite easy.
Namely, the measure of the set which contains all elements of $X$ which have a $0$ in a given $\Z_2$-coordinate has measure $\frac12$.
\footnotetext{For more information on the Haar measure see e.g. \cite{fol95}}

More precisely, let
\begin{itemize}
	\item $1 \leq k_0 \leq k$
	\item $i \in \Z$
	\item $1 \leq n_0 \leq n(M)$
	\item $1 \leq l_0 \leq l$
\end{itemize}
Then
\begin{align*}
	  &\mu (\{~ x \in X = {{((\Z_2^{k})^\Z)}^{n(M)}} \times \Z_2^l ~|~ x_{k_0,i,n_0,l_0} = 0 ~\}) \\
	=~&\mu (\{~ x \in X = {{((\Z_2^{k})^\Z)}^{n(M)}} \times \Z_2^l ~|~ x_{k_0,i,n_0,l_0} = 1 ~\}) \\
	=~&\frac12
\end{align*}
For a set where multiple (say $j$) coordinates are fixed to be $0$ or $1$ each, the measure is $2^{-j}$.

We continue constructing our TDS:
For the division of $X$ we take \IM{\Z_2^{k \cdot n(m)} \times \Z_2^l} to be the index set $J$ of the TDS $M^\prime$ and for
\begin{align*}
	z = (z_1,...,z_n), z_i \in \Z_2^k ~\text{and}~\sigma \in \Z_2^l
\end{align*}
we set
\begin{align*}
	X_{z,\sigma} :=
	\begin{bmatrix}
		\underline{z_{1}} \\
		\vdots \\
		\underline{z_{n}}
	\end{bmatrix} [ \sigma ]
\end{align*}

The sets $X_{z,\INI}$ we devide further into
\begin{align*}
	X_{\overline{z,\INI}} :=
	\begin{bmatrix}
		\DELIM & \EMP & \underline{z_1} \\
		\vdots & \vdots & \underline\vdots \\
		\DELIM & \EMP & \underline{z_n} \\
	\end{bmatrix} [ \INI ]
\end{align*}
and its complement which we call $X_{z,\INI}$

\begin{Remark}
	Each $X_i$ has measure $2^{-(n \cdot k+l)}$ except for those who's state is \INI.
	They have measure $2^{-(3 \cdot n \cdot k+l)}$ if they are of the form $X_{\overline{z,\INI}}$ and the complimental measure otherwise.
\end{Remark}

We define the group $\Gamma$ and its action:
$\Z^{n(m)}$ operates on $X$ via addition on the exponent of the left factor of $X$.
On the right factor we have the natural action of $\Aut \Z_2^l$.
In total, we get
\begin{align*}
	\Gamma := \Z^{n(m)} \times \Aut \Z_2^l
\end{align*}
acting on $X$.
This group action preserves the measure $\mu$ which is a direct implication of the translation invariance of the Haar measure.

Now we set the $\gamma_i$, that is we define an action on $X$ by assigning an element of $\Gamma$ to each $X_{z,\sigma}$.
First notice that
$\Aut Z_2^l$ acts transitively on $\Z_2^l \setminus \{0\}$ (which corresponds to the states of $M$).
To see this, realize that every element of $\Z_2^l \setminus \{0\}$ has order $2$ and thus any mapping \IM{x \mapsto y} for \IM{x,y \in \Z_2^l \setminus \{0\}} can be extended to some (non-unique) automorphism $\phi_{x,y}$.

Now set
\begin{align*}
	\gamma_{z,\sigma} := (T_{direction}(z,\sigma) , \phi_{\sigma,(T_{state}(z,\sigma))}) \in \Gamma
\end{align*}
for each $z \in \Z_2^{k \cdot n(M)}$ and $\sigma \in \Z_2^l \setminus \{0\}$.

Now we only need to set the accepting, rejecting and inital sets:
\begin{align*}
	J_I & := \{(\overline{z,\INI}) ~\vert~ z \in \Z_2^{k \cdot n(M)} \} \\
	J_A & := \{(z,\ACC) ~\vert~ z \in A(M)^{n(M)} \subseteq \Z_2^{k \cdot n(M)} \} \\
	J_R & := \{(z,s) ~\vert~ z \in \Z_2^{k \cdot n(M)} , s = \REJ ~\text{or}~ s \notin S(M) \} \\
\end{align*}
Thereby we have defined a Turing dynamical system $M^\prime$ such that its Turing map $T(M^\prime)$ equals the transition map $T(M)$ of the Turing machine $M$ on its domain which is  \IM{C(M) \subseteq X}.
In the future we will use $T(M^\prime)$ and $T(M)$ interchangeably and call both just $T$ if a precise distinction is not necessary.

Also we have \IM{IC(M) \subseteq I(M^\prime)} up to the \DELIM-symbol.
Equality would of course be nicer here but (regarding $IC(M)$ as a subset of $X$) \IM{\mu(IC(M)) = 0}, since in $IC(M)$ we fix infinitely many symbols to be \EMP.
Having measure $0$ is unacceptable for \IM{I(M^\prime)} and warrants a little discomfort as otherwise Lemma \ref{tm_to_tds:properties:lemma_fundamental_value} would not hold. 
