Let $M$ be a readonly Turing machine.
Then we define the Turing dynamical system $M^\prime$ as follows:

Without loss of generality we may assume that \IM{\left\vert A(M) \right\vert = 2^k} for some \IM{k \in \N}.
Otherwise we add random symbols to the alphabet until it has the desired size.
Similarly we assume that \IM{\left\vert S(M) \right\vert = 2^l-1}.
Now consider the group $\Z_2$.
We set
\begin{align*}
	X := {(\Z_2^{k})^{\Z^{n(M)}}} \times \Z_2^l
\end{align*}
Into this we can embed $C(M)$ by mapping the tapes via the identity to the left factor and the state in any way to the right factor except $0$.

On $Z_2$ we have a probability measure defined by
\begin{align*}
	\mu(0) = \mu(1) = \frac12
\end{align*}
This measure extends to a unique probability measure on X by the Andersen-Jessen theorem. %maybe reference?

%On notation

For the division of $X$ we take \IM{\Z_2^{k * 2 * n(m)} \times \Z_2^l} to be the index set and for
\begin{align*}
	z =
	\begin{bmatrix}
		z_{1,-1} & z_{1,0} \\
		\vdots & \vdots \\
		z_{n,-1} & z_{n,0}
	\end{bmatrix}, z_{i,\pm 1} \in \Z_2^k ~\text{and}~ \sigma \in \Z_2^l
\end{align*}
we set
\begin{align*}
	X_{z,\sigma} :=
	\begin{bmatrix}
		z_{1,-1} & \underline{z_{1,0}} \\
		\vdots & \vdots \\
		z_{n,-1} & \underline{z_{n,0}}
	\end{bmatrix} [ \sigma ]
\end{align*}

\remark Each $X_i$ has measure $2^{-(2*n+l)}$.

We define the group $\Gamma$ and its action:
$\Z^{n(m)}$ operates on the left factor of $X$ via it's natural operation on the index group (which is also $\Z^{n(M)}$).
On the right factor we have the natural action of $\Aut \Z_2^l$.
In total, we get
\begin{align*}
	\Gamma := \Z^{n(m)} \times \Aut \Z_2^l
\end{align*}
acting on $X$.
The group action clearly preserves the measure $\mu$.

As for the $\gamma_i$ notice that
$\Aut Z_2^l$ acts transitively on $\Z_2^l \setminus 0$ (which corresponds to the states of $M$).
To see this, realize that every element of $\Z_2^l \setminus 0$ can be seen as a generator and thus any mapping \IM{x \mapsto y} for \IM{x,y \in \Z_2^l \setminus 0} can be extended to some (non-unique) automorphism $\phi_{x,y}$.

Now set
\begin{align*}
	\gamma_{z,\sigma} := (T_{direction}(z_{*,0},\sigma) , \phi_{\sigma,(T_{state}(z_{*,0},\sigma))}) \in \Gamma
\end{align*}
where \IM{z_{*,0} = (z_{1,0},z_{2,0},...,z_{n,0}) \in (\Z_2^k)^n}

Now we only need to set the accepting, rejecting and inital sets:
\begin{align*}
	J_I & := \{(z,\INI) ~\vert~ z \in \Z_2^{k * 2 * n(M)} , z_{i,-1} = \EMP ~ \forall i \in \{1,...,n\}\} \\
	J_A & := \{(z,\ACC) ~\vert~ z \in \Z_2^{k * 2 * n(M)} \} \\
	J_R & := \{(z,\REJ) ~\vert~ z \in \Z_2^{k * 2 * n(M)} \} \\
\end{align*}
Thereby we have defined a Turing dynamical system $M^\prime$ such that \IM{T(M^\prime) = T(M)} on \IM{C(M) \subseteq X}.
In the future we will use $T(M^\prime)$ and $T(M)$ interchangeably and call both just $T$ if a precise distinction is not necessary.

Also we have \IM{IC(M) \subseteq I(M^\prime)}.
Equality would of course be nicer here but (regarding $IC(M)$ as a subset of $X$) \IM{\mu(IC(M)) = 0} which is unacceptable for \IM{I(M^\prime)} as we will see later and warrants a little discomfort.

We shall now see how properties of $M$ translate to $M^\prime$.
