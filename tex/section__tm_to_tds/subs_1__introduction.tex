Let $M$ be a readonly Turing machine.
Then we define the Turing dynamical system $M^\prime$ as follows:

First of all extend the alphabet of $M$ by the newly introduced $\DELIM$-symbol. This will be required for formal reasons later on. If it is ever encountered on a tape, $M$ should immediately reject.

We will require that \IM{\left\vert A(M)~\cup~\{\DELIM\} \right\vert = 2^k} for some \IM{k \in \N}.
If this is not true, we add random symbols to the alphabet until it has the desired size. By rejecting those symbols whenever they are encountered we basically do not change the behaviour of $M$.
Similarly we may assume that \IM{\left\vert S(M) \right\vert = 2^l-1}.
Now consider the group $\Z_2$.
We set
\begin{align*}
	X := {{((\Z_2^{k})^\Z)}^{n(M)}} \times \Z_2^l
\end{align*}
Into this we can embed the configurations of $M$ by mapping the tapes via the identity to the left factor and the state in any way to the right factor except $0$.
Let us fix any such mapping and regard $C(M)$ as a subset of $X$ from now on.

On $\Z_2$ we have a probability measure defined by
\begin{align*}
	\mu(0) = \mu(1) = \frac12
\end{align*}
This measure extends to the Haar measure\footnotemark on $X$.
%TODO
\footnotetext{Maybe reference and further explanation}
%TODO On notation

For the division of $X$ we take \IM{\Z_2^{k * n(m)} \times \Z_2^l} to be the index set and for
\begin{align*}
	z = (z_1,z_1,...z_n), z_i \in \Z_2^k, \sigma \in \Z_2^l
\end{align*}
we set
\begin{align*}
	X_{z,\sigma} :=
	\begin{bmatrix}
		\underline{z_{1}} \\
		\vdots \\
		\underline{z_{n}}
	\end{bmatrix} [ \sigma ]
\end{align*}

The sets $X_{z,\INI}$ we devide further into
\begin{align*}
	X_{\overline{z,\INI}} :=
	\begin{bmatrix}
		\DELIM & \EMP & \underline{z_1} \\
		\vdots & \vdots & \underline\vdots \\
		\DELIM & \EMP & \underline{z_n} \\
	\end{bmatrix} [ \INI ]
\end{align*}
and its complement which we call $X_{z,\INI}$

\remark Each $X_i$ has measure $2^{-(n+l)}$ except for those who's state is \INI. They have measure $2^{-(3*n+l)}$ if they have are of the form $X_{\overline{z,\INI}}$ and the complimental measure otherwise.

We define the group $\Gamma$ and its action:
$\Z^{n(m)}$ operates on the left factor of $X$ via its natural operation on the exponent of $\Z_2^k$.
On the right factor we have the natural action of $\Aut \Z_2^l$.
In total, we get
\begin{align*}
	\Gamma := \Z^{n(m)} \times \Aut \Z_2^l
\end{align*}
acting on $X$.
This group action preserves the measure $\mu$.
%TODO explain?

As for the $\gamma_i$ notice that
$\Aut Z_2^l$ acts transitively on $\Z_2^l \setminus 0$ (which corresponds to the states of $M$).
To see this, realize that every element of $\Z_2^l \setminus 0$ has order $2$ and thus any mapping \IM{x \mapsto y} for \IM{x,y \in \Z_2^l \setminus 0} can be extended to some (non-unique) automorphism $\phi_{x,y}$.

Now set
\begin{align*}
	\gamma_{z,\sigma} := (T_{direction}(z_{*,0},\sigma) , \phi_{\sigma,(T_{state}(z_{*,0},\sigma))}) \in \Gamma
\end{align*}
where \IM{z_{*,0} = (z_{1,0},z_{2,0},...,z_{n,0}) \in (\Z_2^k)^n}

Now we only need to set the accepting, rejecting and inital sets:
\begin{align*}
	J_I & := \{(\overline{z,\INI}) ~\vert~ z \in \Z_2^{k * n(M)} \} \\
	J_A & := \{(z,\ACC) ~\vert~ z \in A(M)^{k * n(M)} \subseteq \Z_2^{k * n(M)} \} \\
	J_R & := \{(z,s) ~\vert~ z \in \Z_2^{k * n(M)} , s = \REJ \vee s \notin S(M) \} \\
\end{align*}
Thereby we have defined a Turing dynamical system $M^\prime$ such that \IM{T(M^\prime) = T(M)} on \IM{C(M) \subseteq X}.
In the future we will use $T(M^\prime)$ and $T(M)$ interchangeably and call both just $T$ if a precise distinction is not necessary.

Also we have \IM{IC(M) \subseteq I(M^\prime)}.
Equality would of course be nicer here but (regarding $IC(M)$ as a subset of $X$) \IM{\mu(IC(M)) = 0} which is unacceptable for \IM{I(M^\prime)} as we will see in the next Lemma and warrants a little discomfort.
