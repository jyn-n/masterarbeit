\subsection{Properties}
\label{tm_to_tds:properties}

We shall now determine how properties of $M$ translate to $M^\prime$.

First we will see that if $M$ accepts at least one initial element, then $M^\prime$ accepts more than a null-set and vice-versa.
\begin{Lemma}
	\label{tm_to_tds:properties:lemma_fundamental_value}
	$AC(M) \neq \emptyset \Leftrightarrow \Omega_1(M^\prime) > 0$
\end{Lemma}
\proof
	``$\Rightarrow$'':
	Let $(a,\INI) \in AC(M)$.
	First note that $(a,\INI) \in \mathcal{F}_1(M^\prime)$.
	Assume for a moment that $n(M) = 1$, i.e. $M$ has just one tape. Consider
	\begin{align*}
		N := \{~ \sum_{i=1}^k T_{direction}(M)^i(a,\INI)~|~k \in \N ~\}
	\end{align*}
	As $M$ accepts $(a,\INI)$, we have that $(T_{direction}(M)^i(a,\INI))_{i \in \N}$ is constantly $0$ but for finitely many elements
	and $T_{direction}(c) \in \{-1,0,1\}$ for every $c \in C(M)$
	we have that \IM{N = \{ x,x+1,...,y \}} for some \IM{x,y \in \Z} with \IM{x \leq 0}, \IM{y \geq 0}.
	
	Clearly \IM{T^\infty (a,\INI)} only depends on $a_x,...,a_y$, i.e. for \IM{(b,\INI) \in FC(M)} with \IM{b_i = a_i~\text{for every}~ i \in N} we have \IM{T^\infty(a,\INI) = T^\infty(b,\INI)}. Therefore
	\begin{align*}
		\left [ a_x,...,\underline{a_0},...,a_y \right ] \left [\INI \right ] \subseteq \mathcal{F}_1(M^\prime)
	\end{align*}
	But this set has positive measure, which proves the statement.

	If $n(M) > 1$ we can similarly construct a set of positive measure which $M^\prime$ accepts by doing the same construction for each tape.

	``$\Leftarrow$'':
		Consider the following map:
		\begin{align*}
			(\Z_2)^k &\to A(M) \\
			a &\mapsto
			\begin{cases}
				a &\text{, if}~ a \in A(M) \\
				\EMP &\text{, otherwise}
			\end{cases}
		\end{align*}
		This map extends to a map $f: (((\Z_2)^k)^\Z)^{n(M)} \to (A(M)^\Z)^{n(M)}$.
		For any \IM{c = (a,s) \in \F_1(M^\prime)} we have $s = \INI \in S(M)$ and $T^\infty(a,s) = T^\infty(f(a),s)$, since if any symbols outside $A(M)$ were actually relevant, $M^\prime$ would reject $(a,s)$ by construction.

		Now consider the set
		\begin{align*}
			Y := \{ (a,s) \in X ~|~ a_i \neq \EMP ~\forall i \in \Z_+ ~\vee~ a_i \neq \EMP ~\forall i \in \Z_{-} \}
		\end{align*}
		Clearly $\mu(Y) = 0$, so $\mathcal{F}_1 \setminus Y \neq \emptyset$, i.e. there is some $c = (a,s)$ which $M^\prime$ accepts such that there are $x < 0 < y$ with $a_x = a_y = \EMP$.
		But $M$ does not cross the $\EMP$-symbol, so $T^\infty(c) = T^\infty(c^\prime)$ for any $c^\prime \in \lbrack \EMP, a_{x+1}, ... , \underline{a_0}, ... , a_{y-1}, \EMP \rbrack \lbrack s \rbrack$.
		In particular, $M^\prime$ accepts $(f(...,\EMP,a_{x+1},...,\underline{a_0},...,a_{y-1},\EMP,...),s)$.
		
		But that is a finite configuration of $M$ which $M^\prime$ and hence also $M$ accepts, which is exactly what we are looking for.
\endproof

\begin{Lemma}
	\label{tm_to_tds:properties:lemma_foolproof}
	If $M$ is foolproof then so is $M^\prime$
\end{Lemma}
\proof
	Let $c = (a,s) \in X$. Without loss of generality $s \in S(M)$ as otherwise $c$ is immediately rejected by $M^\prime$. So in particular $M^\prime$ stops for $c$ in that case.

	We only consider the case that $M$ has just one tape, otherwise we do the following once for every tape.
	With probability $1$, there are $x < 0 < y$ with $a_x = a_y = \EMP$.
	That is the set of elements which do not meet this criterion is a null-set.
	Since $M$ is readonly, it does not cross the $\EMP$-symbol. Therefore for
	\begin{align*}
		c^\prime = ((...\EMP,a_{x+1},...,a_0,...,a_{y-1},\EMP,...) , s)
	\end{align*} 
	we have $T^\infty(c^\prime) = T^\infty(c)$.

	Now there might be some $i \in \{x+1,...,y-1\}$ such that $a_i \notin A(M)$.
	Either, this symbol is encountered at some point (i.e. $\sum_{i=1}^k T_{direction}^i(c) = i$ for some $k \in \N$), in which case $M^\prime$ rejects $c$.

	Or it is not encountered. But then we may choose an absolutely smaller $x$ or $y$ such that we can construct $c^\prime$ as above and still get $T^\infty(c^\prime) = T^\infty(c)$.
	But if for no \IM{i \in \{x+1,...,y-1\}} we have $a_i \notin A(M)$, then $c^\prime$ is a finite configuration of $M$.
	$M$ stops for this $c^\prime$ because it is foolproof and hence $M^\prime$ stops for $c$.

	To sum up, with probability $1$ either $M^\prime$ rejects $c$ because a dummy symbol or state occurs, or it behaves as $M$ would for $c^\prime$, that is in particular it stops.
\endproof

\begin{Lemma}
	\label{tm_to_tds:properties:lemma_disjoint_accepting_chains}
	$M^\prime$ has disjoint accepting chains.
\end{Lemma}
\proof
	Consider a configuration \IM{c \in F_1}. Call one of its tapes \IM{z \in (\Z_2^k)^\Z}.
	Assume there are \IM{i,j \in \N_0} such that \IM{z_i = z_{-j} = \DELIM}.
	Let $i$ and $j$ be minimal with that property, i.e.
	\begin{align*}
		z \in \left [ \DELIM , z_{-j+1} , ... , \underline{z_0} , ... , z_{i-1} , \DELIM \right ]
	\end{align*}
	where none of the $z_l, l \in \{-j+1,...,i-1\}$ is \DELIM.
	Since $T$ ``shifts the underlinings'' by at most one position and an underlined \DELIM~would be rejected immediately, we know that for preimages of $c$ under several applications of $T$, the underlining must always be on one of $z_{-j+1}$ through $z_{i-1}$. For any initial configuration with this property, the tape is of the form
	\begin{align*}
		\left [ \DELIM , \EMP , \underline{z_{-j+2}} , ... , z_{i-1} , \DELIM \right ]
	\end{align*}
	and everything outside the \DELIM s is uniquely determined by $c$.
	The same argument can be applied to every single tape and the state of an initial configuration is always \INI.
	Thus, if \IM{(T^\infty)^{-1}(\{c\}) \cap I} is not empty, it has exactly one element.

	The assumption that on every tape there is at least one \DELIM~ in both the positive and negative parts is true with probability $1$ and therefore $T^\infty$ is injective on $I$ except for some null-set.
\endproof
