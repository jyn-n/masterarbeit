We shall now see how properties of $M$ translate to $M^\prime$.

\begin{Lemma} \label{tm_to_tds:properties:lemma_measure_finite_configurations}
	$\mu(C(M)) = \mu(FC(M))$
\end{Lemma}
\proof
	This space was unintentionally left blank.
	%TODO
\endproof

\begin{Lemma}
	$AC(M) \neq \emptyset \Leftrightarrow \Omega_1(M^\prime) > 0$
\end{Lemma}
\proof
	``$\Rightarrow$'':
	Let $(a,\INI) \in AC(M)$.
	First note that $(a,\INI) \in \mathcal{F}_1(M^\prime)$.
	Assume for a moment that $n(M) = 1$, i.e. $M$ has just one tape. Consider
	\begin{align*}
		N := \{ \sum_{i=0}^k T_{direction}(M)^i(a,\INI)~|~k \in \N \}
	\end{align*}
	As the sequence $(T^i(M)(a,\INI))_{i \in \N}$ and therefore $(T_{direction}^i(M)(a,\INI))_{i \in \N}$ is constant almost everywhere
	and \IM{|T_{direction}(c)| \leq 1~\forall c \in C(M)}
	we have that \IM{N = \{ x,x+1,...,y \}} for some \IM{x,y \in \Z} with \IM{x \leq 0}, \IM{y \geq 0}.
	
	Clearly \IM{T^\infty (a,\INI)} only depends on $a_x, a_{x+1},...a_y$, i.e. for \IM{(b,\INI) \in FC(M)} with \IM{b_i = a_i~\forall i \in N} we have \IM{T^\infty(M)(a,\INI) = T^\infty(M)(b,\INI)}. Therefore
	\begin{align*}
		\left [ a_x, a_{x+1},...,\underline{a_0},...a_y \right ] \left [\INI \right ] \subseteq \mathcal{F}_1(M^\prime)
	\end{align*}
	But this set has positive measure, which proves the statement.

	If $n(M) > 1$ we can similarly construct a set of positive measure which $M^\prime$ accepts by doing the same construction for each tape.

	``$\Leftarrow$'':
		%TODO
%		In this case we have in particular that $\mathcal{F}_1(M^\prime)$ is not empty.
%		So let $x \in \mathcal{F}_1(M^\prime)$. Doing a similar construction as in the first part of this proof, we can see that only a finite part of $x$ is actually regarded when calculating $T^\infty(x)$.
%		Thus there is a (finite!) initial configuration which $M$ accepts.
\endproof

\begin{Lemma}
	If $M$ is foolproof then so is $M^\prime$
\end{Lemma}
\proof
	By definition we have \IM{X \setminus C(M) \subseteq R} so in particular, $M^\prime$ stops for those configurations.
	If $M$ stops for \IM{x \in C(M)} then so does $M^\prime$ as for those $x$ we have \IM{T^\infty(M)(x) = T^\infty(M^\prime)(x)}
	Since $M$ is foolproof it stops for any finite configuration and so does $M^\prime$. But \IM{\mu(C(M) \setminus FC(M)) = 0} by Lemma \ref{tm_to_tds:properties:lemma_measure_finite_configurations} which completes the proof.
\endproof

\begin{Lemma}
	$M^\prime$ has disjoint accepting chains.
\end{Lemma}
\proof
Consider a configuration \IM{c \in A \cup R} with a tape \IM{z \in (\Z_2^k)^\Z}
Assume there are \IM{i,j \in \N_0} such that \IM{z_i = z_{-j} = \DELIM}.
Let $i$ and $j$ be minimal with that property, i.e.
\begin{align*}
	z \in \left [ \DELIM , z_{-j+1} , ... , \underline{z_0} , ... , z_{i-1} , \DELIM \right ]
\end{align*}
where none of the $z_l$ is \DELIM.
Since $T$ ``shifts the underlinings'' by at most one position and an underlined \DELIM~ would be rejected immediately, we know that for preimages of $c$ under several applications of $T$, the underlining must always be on one of $z_{-j+1}$ through $z_{i-1}$. For any initial configuration with this property, the tape is of the form
\begin{align*}
	\left [ \DELIM , \EMP , \underline{z_{-j+2}} , ... , z_{i-1} , \DELIM \right ]
\end{align*}
and everything outside the \DELIM s is uniquely determined by $c$.
The same argument can be applied to every single tape and the state of an initial configuration is always \INI.
Thus, if \IM{(T^\infty)^{-1}(\{c\}) \cap I} is not empty, it has exactly one element.

The assumption that on every tape there is at least one \DELIM~ in both the positive and negative parts is true with probability $1$ and therefore $T^\infty$ is injective on $I$ except for some null-set.
\endproof
