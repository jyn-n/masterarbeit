\section{Turing dynamical systems}

\begin{Definition}
	A \emph{Turing dynamical system (TDS) } consists of the following data:
	\begin{itemize}
		\item{a probability measure space $(X,\mu)$}
		\item{a discrete countable topological group $\Gamma$}
		\item{a measure preserving right group action $\rho X \curvearrowleft \Gamma$}
		\item{a division $X = \dot\bigcup_{i=1}^n X_i$ into disjoint measurable subsets}
		\item{disjoint sets $J_I, J_R, J_A \subseteq \{1,...,n\}$}
		\item{a map $\{1,...,n\} \to \Gamma, i \mapsto \gamma_i$ such that $\gamma_j = 1~\forall j \in J_R \cup J_A$}
	\end{itemize}
\end{Definition}
From this we can derive the \emph{initial set} $I = \bigcup_{i\in J_I} X_i$, the accepting set $A = \bigcup_{i \in J_A} X_i$ and the rejecting set $R = \bigcup_{i \in J_R} X_i$ as well as the \emph{Turing map}
\begin{align*}
	T_X : X &\to X \\
	x &\mapsto \rho(\gamma_i)(x)&\text{, for}~x \in X_i
\end{align*}
Note how most of a TDS ca be recovered from this derived information.
We lose the measure $\mu$ (but only in so far as $\mu$ is known to be a probability measure which $T_X$ preserves),
those elements of $\Gamma$ that are no $\gamma_i$
and the exact division into $X_i$, but the mapping to the $\gamma_i$ is still known.

\begin{Lemma}
	The Turing map $T_X$ of any TDS is measurable and measure contracting, that is for any measurable $U \subseteq X$ we have that $T_X(U)$ and $T_X^{-1}(U)$ are both measurable and $\mu(T_X(U)) \leq \mu(U)$.
\end{Lemma}
\proof
Let $U \subseteq X$ be measurable. We have
\begin{align*}
	T_X^{-1}(U) = \bigcup_{i=1}^n (\rho(\gamma_i))^{-1}(U)
\end{align*}
and
\begin{align*}
	T_X(U) = \bigcup_{i=1}^n T_X(U \cap X_i) = \bigcup_{i=1}^n \rho(\gamma_i)(U \cap X_i)
\end{align*}
both of which are measurable as all the $X_i$ are measureable and the $\gamma_i$ are measure preserving and thus
\begin{align*}
	&\mu(T_X(U)) \\
	= &\mu(\bigcup_{i=1}^n \rho(\gamma_i)(U \cap X_i)) \\
	\leq &\sum_{i=1}^n \mu(\rho(\gamma_i)(U \cap X_i)) \\
	= &\sum_{i=1}^n \mu(U \cap X_i) \\
	= &\mu(U)
\end{align*}
\endproof

Similar to Turing machines, we define $T_X^\infty(x):=T_X^k(x)$, if $T_X^k(x)=T_X^{k+1}(x)$. Thus we also get the notions of accepting, rejecting and stopping for TDS.
That is, a TDS accepts $x \in X$, if $T_X^\infty(x) \in A$,
rejects it if $T_X^\infty \in R$ 
nd stops for $x$, if it either accepts or rejects it.
We say a TDS is \emph{foolproof} (or \emph{stops on any configuration}, if it stops for all $x \in X$ except for some null set.

The \emph{first fundamental set} of a TDS is defined as
\begin{align*}
	\mathcal{F}_1 :=&\{ x \in I~|~T_X^\infty(x) \in A~\text{and}~x \notin T_X(X) \} \\
	=&(I \cap \bigcup_{i=1}^\infty T_X^{-i}(A)) \setminus T_X(X)
\end{align*}

%second fundamental set, fundamental values

