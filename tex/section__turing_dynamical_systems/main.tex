\section{Turing dynamical systems} \label{tds}

\subsection{Definition}

\begin{Definition}
	A \emph{Turing dynamical system (TDS)} consists of the following data:
	\begin{itemize}
		\item{a probability measure space $(X,\mu)$}
		\item{a discrete countable topological group $\Gamma$}
		\item{a measure preserving group action $\rho : \Gamma \curvearrowright X$ (write $\gamma \bullet x$ for $\rho(\gamma)(x)$)}
		\item{a division $X = \dot\bigcup_{i \in J} X_i$ into finitely many disjoint measurable subsets, $J$ is some finite index set}
		\item{disjoint sets $J_I, J_R, J_A \subseteq J$}
		\item{a map $J \to \Gamma, i \mapsto \gamma_i$ such that $\gamma_j = 1~\text{for all}~ j \in J_R \cup J_A$}
	\end{itemize}
\end{Definition}
%TODO 1=e

From this we can derive the \emph{initial set} $I = \bigcup_{i\in J_I} X_i$, the \emph{accepting set} $A = \bigcup_{i \in J_A} X_i$ and the \emph{rejecting set} $R = \bigcup_{i \in J_R} X_i$ as well as the \emph{Turing map}
\begin{align*}
	T : X &\to X \\
	x &\mapsto \gamma_i \bullet x~\text{, for}~x \in X_i
\end{align*}
Note how most of a TDS can be recovered from this derived information.
We lose the measure $\mu$ (but only in so far as $\mu$ is known to be a probability measure which $T$ preserves),
those elements of $\Gamma$ that are no $\gamma_i$
and the exact division into $X_i$, but we keep the division into preimages of the mapping $x \mapsto i \mapsto \gamma_i$


\begin{Lemma} \label{tds:lemma_tx_measurable:lemma}
	The Turing map $T$ of any TDS is measurable and measure contracting, that is for any measurable $U \subseteq X$ we have that $T(U)$ and $T^{-1}(U)$ are both measurable and $\mu(T(U)) \leq \mu(U)$.

	If $T$ is injective on $U$, then $\mu(T(U)) = \mu(U)$.
\end{Lemma}
\proof
Let $U \subseteq X$ be measurable. We have
\begin{align*}
	T^{-1}(U) = \bigcup_{i=1}^n (\gamma_i^{-1} \bullet U) \cap X_i
\end{align*}
and
\begin{align*}
	T(U) = \bigcup_{i=1}^n T(U \cap X_i) = \bigcup_{i=1}^n \gamma_i \bullet (U \cap X_i)
\end{align*}
both of which are measurable as all the $X_i$ are measureable and the $\gamma_i$ are measure preserving and thus
\begin{align*}
	  &\mu(T(U)) \\
	=~&\mu(\bigcup_{i=1}^n \gamma_i \bullet (U \cap X_i)) \\
	\leq~&\sum_{i=1}^n \mu(\gamma_i \bullet (U \cap X_i)) \\
	=~&\sum_{i=1}^n \mu(U \cap X_i) \\
	=~&\mu(U)
\end{align*}
In case of injectivity, the ``$\leq$'' is in fact an equality.
\endproof


The \emph{first fundamental set} of a TDS is defined as
\begin{align*}
	\mathcal{F}_1 :=&\{ x \in I~|~T_X^\infty(x) \in A~\text{and}~x \notin T_X(X) \} \\
	=&(I \cap \bigcup_{i=1}^\infty T_X^{-i}(A)) \setminus T_X(X)
\end{align*}
That is all the initial configurations the TDS accepts, that cannot be reached from any other configuration.
The \emph{second fundamental set} is
\begin{align*}
	\mathcal{F}_2 :=&T_X^\infty(\mathcal{F}_1) \\
	=&\bigcup_{i=1}^\infty T_X^i( I \cap T_X^{-i}(A) \ T_X(X))
\end{align*}
Using Lemma \ref{tds:lemma_tx_measurable:lemma} it is easy to see that both fundamental sets are measurable.
Their measures are called the \emph{first} and \emph{second fundamental value} and denoted by $\Omega_i$.

