We will now see some properties of TDS that will become important later on.

Similar to Turing machines, we define $T^\infty(x):=T^k(x)$, if $T^k(x)=T^{k+1}(x)$ and leave it undefined if no such $k$ exists.

From this we also get the notions of accepting and rejecting for a TDS.
That is, a TDS accepts $x \in X$, if $T^\infty(x) \in A$,
rejects it if $T^\infty(x) \in R$ 
and stops for $x$, if it either accepts or rejects it.

\begin{Definition}
	The \emph{first fundamental set} of a TDS is defined as
	\begin{align*}
		\F_1 :=&~\{ x \in I~|~T^\infty(x) \in A~\text{and}~x \notin T(X) \} \\
		=&~(I \cap \bigcup_{i=1}^\infty T^{-i}(A)) \setminus T(X)
	\end{align*}
	That is all the initial configurations the TDS accepts, that cannot be reached from any other configuration.
	The \emph{second fundamental set} is
	\begin{align*}
		\F_2 :=&~T^\infty(\mathcal{F}_1) \\
		=&~\bigcup_{i=1}^\infty T^i( I \cap T^{-i}(A) \setminus T(X))
	\end{align*}
	Using Lemma \ref{tds:lemma_tx_measurable:lemma} it is easy to see that both fundamental sets are measurable.
	Their measures are called the \emph{first} and \emph{second fundamental value} and denoted by $\Omega_i$.
\end{Definition}

%TODO

\begin{Definition}
	\
	\begin{itemize}
		\item We call a TDS \emph{foolproof} (or say it \emph{stops on any configuration}), if it stops for all $x \in X$ except for some null set.

		\item If $T^\infty$ is injective on $F_1$ except for some null set, we say that the TDS has $\emph{disjoint accepting chains}$.

		\item If \IM{\mu(~T(X) \cap I~) = 0}, the TDS \emph{does not restart}.
	\end{itemize}
\end{Definition}
